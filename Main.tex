% LaTeX source for textbook ``How to think like a computer scientist''
% Copyright (C) 1999  Allen B. Downey
% Copyright (C) 2009  Thomas Scheffler
% Copyright (C) 2023  Michael Penta

\documentclass[a4paper]{book}
\usepackage[ngerman,english]{babel}
\usepackage[latin1]{inputenc}
\usepackage[T1]{fontenc}
\usepackage{lmodern}
\usepackage{epsfig}
\usepackage{imakeidx}
\usepackage{url}
\usepackage{fancyhdr}
\usepackage{multicol}
\usepackage[hidelinks]{hyperref}
\usepackage{longtable}
\usepackage{ifthen}
\usepackage{boxedminipage}
\usepackage{todonotes}
% conditional compilation of the document for different languages


%\newboolean{German}
%\setboolean{German}{false}

% the exercise environment

\newcounter{exercisenum}                                                  
     
% by default, the exercise number includes the chapter number             
% this way, an exercise label is a complete, unique exercise id           
     
\renewcommand{\theexercisenum}{{\thechapter}.\arabic{exercisenum}}  

% Standard font size for exercise/problem text                            

\newenvironment{exercisesize}{\begin{small}}{\end{small}}                 

\newcommand{\exerciseheader}[2]{                                          
     
  \begin{exercisesize}                                                    
     
  % Use alphabetic chars for subparts of exercises,                            
  % and roman numerals for subparts of them.
     
  \def\theenumi{\alph{enumi}}                                             
  \def\labelenumi{\theenumi.}                                             
  \def\theenumii{\roman{enumii}}                                          
  \def\labelenumii{\theenumii.}                                           
  {\bf Exercise {#1}{#2}}\hspace{0.1in}                 
}                                                                         

\newcommand{\startexercise}[1]{%
  \refstepcounter{exercisenum}                                            
  \exerciseheader{\theexercisenum}{#1}                                    
}                                                                         

\newcommand{\stopexercise}{%                                                   
  {\hfill}                                                               
  \end{exercisesize}      
}                                                         
     
\newcommand{\normaldif}{}                                                 
     
\newcommand{\bigdif}{\dag{}}                                              
     
\newcommand{\verybigdif}{\ddag{}}             

\newenvironment{exercise}{\startexercise{\normaldif{}}}{\stopexercise}    
     
\newenvironment{hardexercise}{\startexercise{\bigdif{}}}{\stopexercise}   
     
%% end of the exercise environment


%%------------------------------------------------------------
% formatting commands

\sloppy
\setlength{\topmargin}{0.125in}
\setlength{\oddsidemargin}{0.875in}
\setlength{\evensidemargin}{0.875in}

\setlength{\headsep}{3ex}
\setlength{\textheight}{8in}

\setlength{\parindent}{0.0in}
\setlength{\parskip}{1.7ex plus 0.5ex minus 0.5ex}
\renewcommand{\baselinestretch}{1.02}

% see LaTeX Companion page 62
\setlength{\topsep}{-0.0\parskip}
\setlength{\partopsep}{-0.5\parskip}
\setlength{\itemindent}{0.0in}
\setlength{\listparindent}{0.0in}

% see LaTeX Companion page 26
% these are copied from /usr/local/teTeX/share/texmf/tex/latex/base/book.cls
% all I changed is afterskip

\makeatletter
\renewcommand{\section}{\@startsection 
    {section} {1} {0mm}%
    {-3.5ex \@plus -1ex \@minus -.2ex}%
    {0.7ex \@plus.2ex}%
    {\normalfont\Large\bfseries}}
\renewcommand\subsection{\@startsection {subsection}{2}{0mm}%
    {-3.25ex\@plus -1ex \@minus -.2ex}%
    {0.3ex \@plus .2ex}%
    {\normalfont\large\bfseries}}
\renewcommand\subsubsection{\@startsection {subsubsection}{3}{0mm}%
    {-3.25ex\@plus -1ex \@minus -.2ex}%
    {0.3ex \@plus .2ex}%
    {\normalfont\normalsize\bfseries}}

\makeatother

\newcommand{\beforeverb}{\vspace{0.6\parskip}}
\newcommand{\afterverb}{\vspace{0.6\parskip}}

\newcommand{\adjustpage}[1]{\enlargethispage{#1\baselineskip}}
\newcommand{\clearemptydoublepage}{\newpage{\pagestyle{empty}\cleardoublepage}}
\newcommand{\blankpage}{\pagestyle{empty}\vspace*{1in}\newpage}

\newcommand{\beforefig}{\vspace{1.3\parskip}}
\newcommand{\afterfig}{\vspace{-0.2\parskip}}
\newcommand{\myfig}[1]{
    \beforefig
    \centerline{\epsfig{#1,scale=0.8}}
    \afterfig
}

\newcommand{\beforechapter}{
%    \clearemptydoublepage 
    \cleardoublepage 
    \setcounter{exercisenum}{0}
}

\pagestyle{fancyplain}

\renewcommand{\chaptermark}[1]{\markboth{#1}{}}
\renewcommand{\sectionmark}[1]{\markright{\thesection\ #1}{}}

\lhead[\fancyplain{}{\bfseries\thepage}]%
      {\fancyplain{}{\bfseries\rightmark}}
\rhead[\fancyplain{}{\bfseries\leftmark}]%
      {\fancyplain{}{\bfseries\thepage}}
\cfoot{}

% turn off the rule under the header
%\setlength{\headrulewidth}{0pt}

% the following is a brute-force way to prevent the headers
% from getting transformed into all-caps
\renewcommand\MakeUppercase{}


\sloppy
\setlength{\topmargin}{0.75in}
\setlength{\headsep}{0.5in}
\setlength{\oddsidemargin}{1.0in}
\setlength{\evensidemargin}{.95in}
\makeindex


%%-----------------------------------------------------------
% beginning of the document

\begin{document}

\thispagestyle{empty}

\begin{flushright}
\vspace*{2.5in}

{\huge How to Think Like a Computer Scientist}

\vspace{0.25in}

{\LARGE C Version}

\vspace{1in}

{\Large Michael Penta}

{based on previous work by Allen B. Downey and Thomas Scheffler}

\vspace{1in}

{\small October  1, 2023}
\vfill

\end{flushright}



Copyright (C) 1999  Allen B. Downey\\
Copyright (C) 2009  Thomas Scheffler\\
Copyright (C) 2023 Michael Penta\\

\vspace{0.25in}

Permission is granted to copy, distribute, transmit and adapt this
work under the Creative Commons Attribution-NonCommercial-ShareAlike 4.0
International License: \url{https://creativecommons.org/licenses/by-nc/4.0/}.

If you are interested in distributing a commercial version of this
work, please contact the author(s).

The \LaTeX\ source and code for this book is available from: \\
\url{https://github.com/mkpenta/ThinkC}

\frontmatter
\tableofcontents


\mainmatter
\include{Chapter1}
% LaTeX source for textbook ``How to think like a computer scientist''
% Copyright (C) 1999  Allen B. Downey
% Copyright (C) 2009  Thomas Scheffler
% Copyright (C) 2023  Michael Penta

\setcounter{chapter}{1}
\chapter{Variables and types}

\section{More output}
\index{output}
\index{statement!output}

As I mentioned in the last chapter, you can put as many statements as
you want in {\tt main()}.  For example, to output more than one line:

\begin{verbatim}

  #include <stdio.h>
  #include <stdlib.h>

  /* main: generate some simple output */

  int main (void)
  {
        printf ("Hello World.\n");		    /* output one line */
        printf ("How are you?\n");		    /* output another line */       
        return (EXIT_SUCCESS);
  }

\end{verbatim}
%
As you can see, it is legal to put comments at the
end of a line, as well as on a line by themselves.

\index{String}
\index{type!String}

The phrases that appear in quotation marks are called {\bf strings},
because they are made up of a sequence (string) of letters.  Actually,
strings can contain any combination of letters, numbers, punctuation
marks, and other special characters.

\index{newline}

Often it is useful to display the output from multiple output
statements all on one line.  You can do this by leaving out
the {\tt $\backslash$n} from the first {\tt printf}:

\begin{verbatim}
  int main (void)
  {
        printf ("Goodbye, ");
        printf ("cruel world!\n");	     
        return (EXIT_SUCCESS);
  }
\end{verbatim}
%
In this case the output appears on a single line as
{\tt Goodbye, cruel world!}.  Notice that there is a space
between the word ``Goodbye,'' and the second quotation mark.
This space appears in the output, so it affects the behavior
of the program.

Spaces that appear outside of quotation marks generally do
not affect the behavior of the program.  For example, I
could have written:

\begin{verbatim}
  int main(void)
  {
  printf("Goodbye, ");
  printf("cruel world!\n");	     
  return(EXIT_SUCCESS);
  }
\end{verbatim}
%
This program would compile and run just as well as the original.
The breaks at the ends of lines (newlines) do not affect
the program's behavior either, so I could have written:

\begin{verbatim}
  int main(void){printf("Goodbye, ");printf("cruel world!\n");
  return(EXIT_SUCCESS);}
\end{verbatim}
%
That would work, too, although you have probably noticed that
the program is getting harder and harder to read.  Newlines and
spaces are useful for organizing your program visually, making
it easier to read the program and locate syntax errors.

\section{Values}
\index{value}
\index{type}

Computer programs operate on values stored in 
computer memory.
A value ---like a letter or
a number--- is one of the fundamental things that a program manipulates.  
The only values we have
manipulated so far are the strings we have been outputting, like
{\tt "Hello, world."}.  You (and the compiler) can identify
these string values because they are enclosed in quotation marks.
%constant values

There are different kinds of values, including integers and characters.
It is important for the program to know exactly what kind of value
is manipulated because not all manipulations will make sense on all
values.
We therefore distinguish between different {\bf types} of values.  
%Example 'a' + 'a'

An integer is a whole number like 1 or 17.  You can output
integer values in a similar way as you output strings:

\begin{verbatim}
   printf("%i\n", 17);
\end{verbatim}
%
When we look at the \texttt{printf()} statement more closely, we
notice that the value we are outputting no longer appears
inside the quotes, but behind them separated by comma.
The string is still there, but now contains a {\tt \%i} instead of
any text. 
The {\tt \%i} a placeholder that tells the \texttt{printf()} command
to print an integer value (you can also use {\tt \%d} for integers). 
Several such placeholders, called {\bf format specifiers}, exist
for different data types and formatting options of the output. 
We can look at more next.

A character value is a letter or digit or punctuation mark
enclosed in single quotes, like {\tt 'a'} or {\tt '5'} - that is the character 5 not the integer 5.
You can output character values in a similar way:

\begin{verbatim}
   printf("%c\n", '}');
\end{verbatim}
%
This example outputs a single closing curly-bracket on a line
by itself. It uses the {\tt \%c} placeholder to signify the output of a character
value.

It is easy to confuse different types of values, like {\tt "5"}, {\tt
'5'} and {\tt 5}, but if you pay attention to the punctuation, it
should be clear that the first is a string, the second is a character
and the third is an integer.  The reason this distinction is important
should become clear soon.

\section {Variables}
\index{variable}
\index{value}

One of the most powerful features of a programming language is the
ability to manipulate values through the use of {\bf variables}.  So far
the values that we have used in our statements where fixed to what 
was written in the statement. Now we will use a variable as a named 
location that stores a value.  

Just as there are different types of values (integer, character,
etc.), there are different types of variables.  When you create a new
variable, you have to declare what type it is.  For example, the
character type in C is called {\tt char}.  The following statement
creates a new variable named {\tt fred} that has type {\tt char}.

\begin{verbatim}
    char fred;
\end{verbatim}
%
This kind of statement is called a {\bf declaration}.

The type of a variable determines what kind of values it can
store.  A {\tt char} variable can contain characters, and it should
come as no surprise that {\tt int} variables can store integers.

Contrary to other programming languages, C does not have a 
dedicated variable type for the storage of string values. We will see in
a later chapter how string values are stored in C. 
%but we
%are going to skip that for now (see Chapter~\ref{strings}).

\index{declaration}
\index{statement!declaration}

To create an integer variable, the syntax is 

\begin{verbatim}
    int bob;
\end{verbatim}
%
where {\tt bob} is the arbitrary name you choose to identify the
variable.  In general, you will want to make up variable names
that indicate what you plan to do with the variable.  For
example, if you saw these variable declarations:

\begin{verbatim}
    char first_letter;
    char last_letter;
    int hour, minute;
\end{verbatim}
%
you could probably make a good guess at what values
would be stored in them.  This example
also demonstrates the syntax for declaring multiple variables
with the same type: {\tt hour} and {\tt minute}
are both integers ({\tt int} type).

ATTENTION: The older C89 standard allows variable declarations
only at the beginning of a block of code. It is therefore necessary
to put variable declarations before any other statements,
even if the variable itself is only needed much later in your program.

\section{Assignment}
\index{assignment}
\index{statement!assignment}

Now that we have created some variables, we would like to
store values in them.  We do that with an {\bf assignment
statement}.

\begin{verbatim}
    first_letter = 'a';   /* give first_letter the value 'a' */
    hour = 11;            /* assign the value 11 to hour */
    minute = 59;          /* set minute to 59 */
\end{verbatim}
%
This example shows three assignments, and the comments show
three different ways people sometimes talk about assignment
statements.  The vocabulary can be confusing here, but the
idea is straightforward:

\begin{itemize}

\item When you declare a variable, you create a named storage location.

\item When you make an assignment to a variable, you give it a value.

\end{itemize}

A common way to represent variables on paper is to draw a box
with the name of the variable on the outside and the value
of the variable on the inside.  This kind of figure is called
a {\bf state diagram} because is shows what state each 
variable is in (you can think of it as the variable's ``state of
mind'').
This diagram shows the effect of the three assignment statements:

%\vspace{0.1in}
%\centerline{\epsfig{figure=figs/assign.eps}}
%\vspace{0.1in}

\setlength{\unitlength}{1mm}
\begin{picture}(20,17)
\put(7,12){\large \texttt{first\_letter}}
\put(46,12){\large \texttt{hour}}
\put(74,12){\large \texttt{minute}}
\put(10,0){\framebox(20,10){{\large \textsf{a}}}}
\put(40,0){\framebox(20,10){{\large \textsf{11}}}}
\put(70,0){\framebox(20,10){{\large \textsf{59}}}}
\end{picture}

When we assign values to variables, we have to make sure that
the assigned value corresponds to the type of the variable.
In C  a variable has to have the same type as the
value you assign.  For example, you cannot store a string in
an {\tt int} variable.  The following statement generates a compiler
warning:

\begin{verbatim}
    int hour;
    hour = "Hello.";       /* WRONG !! */
\end{verbatim}
%
This rule is sometimes a source of confusion, because there are many
ways that you can convert values from one type to another, and C
sometimes converts things automatically.  But for now you should
remember that as a general rule variables and values have the same
type, and we'll talk about special cases later.

Another source of confusion is that some strings {\em look}
like integers, but they are not.  For example,
the string {\tt "123"}, which is made up of the
characters {\tt 1}, {\tt 2} and {\tt 3}, is not
the same thing as the {\em number} {\tt 123}.
This assignment is illegal:

\begin{verbatim}
    minute = "59";         /* WRONG!! */
\end{verbatim}
%
\section{Outputting variables}
\label{output variables}

You can output the value of a variable using the same commands
we used to output simple values.

\begin{verbatim}

    int hour, minute;
    char colon;

    hour = 11;
    minute = 59;
    colon = ':';

    printf ("The current time is ");
    printf ("%i", hour);
    printf ("%c", colon);
    printf ("%i", minute);
    printf ("\n"); 
  
\end{verbatim}
%
This program creates two integer variables named {\tt hour} and {\tt
minute}, and a character variable named {\tt colon}.  It assigns
appropriate values to each of the variables and then uses a series
of output statements to generate the following:

\begin{verbatim}
    The current time is 11:59
\end{verbatim}

When we talk about ``outputting a variable,'' we mean outputting the
{\em value} of the variable.  The name of a variable only has significance for
the programmer. The compiled program no longer contains a human readable
reference to the variable name in your program. 

The \texttt{ printf()} command is capable of outputting several variables
in a single statement. To do this, we need to put placeholders
in the so called \emph{format string}, that indicate the position where the variable value will
be put. The variables will be inserted in the order of their appearance in 
the statement. It is important to observe the right order and type for the variables.

By using a single output statement, we can make the previous program more
concise:

\begin{verbatim}
    
    int hour, minute;
    char colon;

    hour = 11;
    minute = 59;
    colon = ':';

    printf ("The current time is %i%c%i\n", hour, colon, minute);
    
\end{verbatim}
%
On one line, this program outputs a string, two integers and a character.  Very impressive!

\section{Keywords}
\index{keyword}

A few sections ago, I said that you can make up any name you
want for your variables, but that's not quite true.  There
are certain words that are reserved in C because they are
used by the compiler to parse the structure of your program,
and if you use them as variable names, it will get confused.
These words, called {\bf keywords}, include {\tt int},
{\tt char}, {\tt void} and many more.

\vskip 1em

\setlength{\fboxsep}{6pt} 
\begin{center}
\begin{boxedminipage}[c]{.9\linewidth}
\begin{center}
\begin{multicols}{5}[\underline{Reserved keywords in the C language}]
\begin{verbatim}
auto 
break 
case 
char 
const 
continue 
default 
do 
double 
else 
enum 
extern 
float 
for 
goto 
if 
inline 
int 
long 
register 
restrict 
return 
short 
signed 
sizeof 
static 
struct 
switch 
typedef 
union 
unsigned 
void 
volatile 
while 
_Bool 
_Complex 
_Imaginary 
\end{verbatim}
\end{multicols}
\end{center}
\end{boxedminipage}
\end{center}

\vskip 1em
The complete list of keywords is included in the C Standard, which
is the official language definition adopted by the the International
Organization for Standardization (ISO) on September 1, 1998.  

%You can download a copy electronically from
%
%\begin{verbatim}
%    http://www.ansi.org/
%\end{verbatim}
%
Rather than memorize the list, I would suggest that you
take advantage of a feature provided in many development
environments: code highlighting.  As you type, different
parts of your program should appear in different colors.  For
example, keywords might be blue, strings red, and other code
black.  If you type a variable name and it turns blue, watch
out!  You might get some strange behavior from the compiler.

\section{Operators}
\label{operators}
\index{operator}


{\bf Operators} are special symbols that are used to represent
simple computations like addition and multiplication.  Most
of the operators in C do exactly what you would expect them
to do, because they are common mathematical symbols.  For
example, the operator for adding two integers is {\tt +}.

The following are all legal C expressions whose meaning is
more or less obvious:

\begin{verbatim}
    1+1        hour-1       hour*60+minute     minute/60
\end{verbatim}
%
{\bf Expressions} can contain both variables
names and values.  In each case the name of the variable is
replaced with its value before the computation is performed.

\index{expression}

Addition, subtraction and multiplication all do what you
expect, but you might be surprised by division.  For example,
the following program:

\begin{verbatim}

  int hour, minute;
  hour = 11;
  minute = 59;
  printf ("Number of minutes since midnight: %i\n", hour*60 + minute);
  printf ("Fraction of the hour that has passed: %i\n", minute/60);

\end{verbatim}
%
would generate the following output:

\begin{verbatim}
    Number of minutes since midnight: 719
    Fraction of the hour that has passed: 0
\end{verbatim}
%
The first line is what we expected, but the second line is
odd.  The value of the variable {\tt minute} is 59, and
59 divided by 60 is 0.98333, not 0.  The reason for the
discrepancy is that C is performing {\bf integer division}.

\index{type!int}
\index{integer division}
\index{arithmetic!integer}
\index{division!integer}
\index{operand}

When both of the {\bf operands} are integers (operands are the things
operators operate on), the result must also be an integer,
and by definition integer division always rounds {\em down},
even in cases like this where the next integer is so close.

A possible alternative in this case is to calculate a percentage
rather than a fraction:

\begin{verbatim}
    printf ("Percentage of the hour that has passed: ");
    printf ("%i\n", minute*100/60);
\end{verbatim}
%
The result is:

\begin{verbatim}
    Percentage of the hour that has passed: 98
\end{verbatim}
%
Again the result is rounded down, but at least now the answer
is approximately correct.  In order to get an even more accurate
answer, we could use a different type of variable, called
floating-point, that is capable of storing fractional values.
We'll get to that in the next chapter.

\section{Order of operations}
\index{precedence}
\index{order of operations}

When more than one operator appears in an expression the order
of evaluation depends on the rules of {\bf precedence}.  A
complete explanation of precedence can get complicated, but
just to get you started:

\begin{itemize}

\item Multiplication and division happen before
addition and subtraction.  So {\tt 2*3-1} yields 5, not 4, and {\tt
2/3-1} yields -1, not 1.
%(remember that in integer division {\tt 2/3} is 0).

\item If the operators have the same precedence they are evaluated
from left to right.  So in the expression {\tt minute*100/60},
the multiplication happens first, yielding {\tt 5900/60}, which
in turn yields {\tt 98}.  If the operations had gone from right
to left, the result would be {\tt 59*1} which is {\tt 59}, which
is wrong.

\item Any time you want to override the rules of precedence (or
you are not sure what they are) you can use parentheses.  Expressions
in parentheses are evaluated first, so {\tt 2*(3-1)} is 4.
You can also use parentheses to make an expression easier to
read, as in {\tt (minute*100)/60}, even though it doesn't
change the result.

\end{itemize}

\section{Operators for characters}
\index{character operator}
\index{operator!character}

Interestingly, the same mathematical operations that work on
integers also work on characters.  For example,

\begin{verbatim}
    char letter;
    letter = 'a' + 1;
    printf ("%c\n", letter);
\end{verbatim}
%
outputs the letter {\tt b}.  Although it is syntactically legal
to multiply characters, it is almost never useful to do it.

Earlier I said that you can only assign integer values to
integer variables and character values to character variables,
but that is not completely true.  In some cases, C converts
automatically between types.  For example, the following is
legal.

\begin{verbatim}
    int number;
    number = 'a';
    printf ("%i\n", number);
\end{verbatim}
%
The result is 97, which is the number that is used internally
by C to represent the letter {\tt 'a'}.  However, it is
generally a good idea to treat characters as characters, and
integers as integers, and only convert from one to the other
if there is a good reason.

Automatic type conversion is an example of a common problem in designing a
programming language, which is that there is a conflict between {\bf
formalism}, which is the requirement that formal languages should have
simple rules with few exceptions, and {\bf convenience}, which is the
requirement that programming languages be easy to use in practice.

More often than not, convenience wins, which is usually good for
expert programmers, who are spared from rigorous but unwieldy
formalism, but bad for beginning programmers, who are often baffled
by the complexity of the rules and the number of exceptions.  In this
book I have tried to simplify things by emphasizing the rules and
omitting many of the exceptions.


\section{Composition}
\index{composition}
\index{expression}

So far we have looked at the elements of a programming
language---variables, expressions, and statements---in
isolation, without talking about how to combine them.

One of the most useful features of programming languages
is their ability to take small building blocks and
{\bf compose} them.  For example, we know how to multiply
integers and we know how to output values; it turns out we can
do both at the same time:

\begin{verbatim}
    printf ("%i\n", 17 * 3);
\end{verbatim}
%
Actually, I shouldn't say ``at the same time,'' since in reality
the multiplication has to happen before the output, but
the point is that any expression, involving numbers, characters,
and variables, can be used inside an output statement.  We've
already seen one example:

\begin{verbatim}
    printf ("%i\n", hour * 60 + minute);
\end{verbatim}
%
You can also put arbitrary expressions on the right-hand
side of an assignment statement:

\begin{verbatim}
    int percentage;
    percentage = (minute * 100) / 60;
\end{verbatim}
%
This ability may not seem so impressive now, but we will see
other examples where composition makes it possible
to express complex computations neatly and concisely.

WARNING: There are limits on where you can use certain
expressions; most notably, the left-hand side of an assignment
statement has to be a {\em variable} name, not an expression.
That's because the left side indicates the storage location
where the result will go.  Expressions
do not represent storage locations, only values.  So the
following is illegal:  {\tt minute + 1 = hour;}.


\section{Scanning User Input}
\index{scanning}
\index{input}
In all of the examples so far we have assigned variable values before we run our program. 
We can also ask the user to input values when the program is running. 
 {\tt scanf()} is a function in the C programming language that is used to read input from the user.
  It can be used to read various types of data, such as integers, characters, and strings (we will get back to strings in a later chapter)
 
To read a single character from the user, you can use the  {\tt \%c} 
format specifier. The {\tt \%c} format specifier tells {\tt scanf()} to read a 
character and store it in the variable. For example:
\begin{verbatim}
char c;
scanf("%c", &c);
\end{verbatim}
%

In this example, {\tt scanf()} will read a character from the user's input and store it in the variable c. Note that the {\tt \&} operator is used when we refer to the variable name. This will be explained in a later chapter, but it is important not to forget it here.

To read an integer, you can use the  {\tt \%d} or  {\tt \%i}  format specifier. These specifiers tells {\tt scanf()} to read an integer and store it in the variable.  For example:
\begin{verbatim}
int x;
scanf("%d", &x);
\end{verbatim}
In this example, {\tt scanf()} will read an integer from the user's input and store it in the variable x.

{\tt scanf()} can also be used to scan strings, but we will get back to strings in a later chapter.


\section{Prompting for User Input}
\index{scanning}
\index{input}
\index{puts}

Prompting for input is a way to gather information from the user and store it in variables for use in the program. 
When using  {\tt scanf()}, you first need to print a prompt to let the user know what to enter. This can be done using {\tt printf}. 

For example, to prompt the user for an integer, you can use:

\begin{verbatim}
	int x;
	printf("%s", "Enter an integer: \n");
	scanf("%d", &x);
\end{verbatim}
%
In this example, the {\tt printf()} function is used to print the message "Enter an integer: "  
with a new line at the end. The user can enter an integer and the {\tt scanf()}  
function is used to read the integer from the user and store it in the variable x.

{\bf Caution} It is considered insecure to use{\tt printf()} to print a string without the string format specifier.  

\begin{verbatim}
	printf( "Enter an integer: \n");  //works but is consdered insecure
	printf("%s", "Enter an integer: \n");  // this is considered a secure use
\end{verbatim}
%

As an alternative to printing strings with {\tt printf()},  you can use the {\tt puts()} function to print a string. 
Unlike {\tt printf()}, {\tt puts()} automatically appends a newline character after the string, 
which can be useful if you want to print multiple strings on separate lines.

For example:
\begin{verbatim}
puts("Welcome to the program");
puts("Enter an integer:");
\end{verbatim}
%
\section{Mixing Calls to Scan Integers and Character}
\index{scanning}
\index{input}
It is important to note that when mixing calls to {\tt scanf()} to read different types, newline characters can cause problems. 
For example, if the user enters an integer followed by a newline character, the newline character will be 
left in the input buffer and can cause issues when trying to read the next input using scanf(). 

One way to manage this is to include a space in the quotes before the format specifier - like this {\tt " \%d"}. 
The space indicates to ignore whitespace characters.

\begin{verbatim}
	
	#include <stdio.h>
	#include <stdlib.h>
	
	int main (void)
	{
			int age;
			char initial;
			puts("enter your age");
			scanf("%d", &age);
			puts("enter your first intial");
			scanf(" %c", &initial); //note the space after " and before %
	}
	
\end{verbatim}
%
 Another way to manage this is to insert a second call to scan the newline.

\begin{verbatim}
	
	#include <stdio.h>
	#include <stdlib.h>
	
	int main (void)
	{
			int age;
			char initial;
			char newline;
   			puts("enter your age");
			scanf("%d", &age);
			scanf("%c", &newline); //this will scan the extra new line
			puts("enter your first intial");
			scanf("%c", &initial); 
	}
	
\end{verbatim}
%

\section{Glossary}

\begin{description}

\item[variable:] A named storage location for values.  All
variables have a type, which determines which values it can
store.

\item[value:] A letter, or number, or other thing that can be
stored in a variable.  

\item[type:] The meaning of values.  The types
we have seen so far are integers ({\tt int} in C) and characters ({\tt
char} in C).

\item[keyword:]  A reserved word that is used by the compiler
to parse programs.  Examples we have seen include {\tt int},
{\tt void} and {\tt char}.

\item[statement:] A line of code that represents a command or
action.  So far, the statements we have seen are declarations,
assignments, and output statements.

\item[declaration:] A statement that creates a new variable and
determines its type.

\item[assignment:] A statement that assigns a value to a variable.

\item[expression:] A combination of variables, operators and
values that represents a single result value.  Expressions also
have types, as determined by their operators and operands.

\item[format specifier:] A special character or sequence of characters that tells the printing and scanning functions how to format and interpret data.

\item[operator:] A special symbol that represents a simple
computation like addition or multiplication.

\item[operand:] One of the values on which an operator operates. 

\item[precedence:] The order in which operations are evaluated.

\item[composition:] The ability to combine simple
expressions and statements into compound statements and expressions
in order to represent complex computations concisely.

\index{variable}
\index{value}
\index{type}
\index{keyword}
\index{statement}
\index{assignment}
\index{expression}
\index{operator}
\index{operand}
\index{composition}
\index{scanning}
\index{format}


\end{description}


\section{Exercises}
\setcounter{exercisenum}{0}

\input{exercises/Exercise_2_english}





% LaTeX source for textbook ``How to think like a computer scientist''
% Copyright (C) 1999  Allen B. Downey
% Copyright (C) 2009  Thomas Scheffler
% Copyright (C) 2023  Michael Penta

\setcounter{chapter}{2}
\chapter{Function}

\section{Floating-point}
\index{floating-point number}
\index{type!double}
\index{double (floating-point)}

In the last chapter we had some problems dealing with numbers
that were not integers.  We worked around the problem by measuring
percentages instead of fractions, but a more general solution is
to use floating-point numbers, which can represent fractions
as well as integers.  In C, there are two floating-point types,
called {\tt float} and {\tt double}.  These two different data types
are  used to store decimal numbers. A "float" is a single-precision number, 
which means it has less precision (or fewer digits) than a "double", 
which is a double-precision number. Because of this,
a double is more accurate when dealing with decimal numbers than a float. 
The trade-off is precision for memory usage - the more precise data type uses more bits
to store the value. In this text we accept this trade-off  and exclusively use the
{\tt double} data type for floating point numbers.  


You can create floating-point variables and assign values to them
using the same syntax we used for the other types.  For example:

\begin{verbatim}
  double pi;
  pi = 3.14159;
\end{verbatim}
%

To print a single precision {\tt float}, we can use the format specifier {\tt \%f}, 
but for a double we need to use {\tt \%lf} -- that is L F, as in Long Float.  
In C, the type modifier {\tt long} doubles the amount of memory (if possible) for a data type. 
If we have a 4 byte {\tt int}, then a {\tt long int} would be 8 bytes. 
We can also reduce the amount of memory allocated with the type modifier {\tt short}.
Our 4 byte {\tt int}, is reduced to 2 bytes when we use a {\tt a short int}.
In essence our {\tt double} data type is a {\tt long float}, hence the {\tt \%lf}.

\begin{verbatim}
	double pi;
	pi = 3.14159;
	printf("%lf\n", pi);
\end{verbatim}
%

It is also legal to declare a variable and assign a value to it at the
same time:

\begin{verbatim}
  int x = 1;
  char first_char = "a";
  double pi = 3.14159;
\end{verbatim}
%
In fact, this syntax is quite common.  A combined declaration
and assignment is sometimes called an {\bf initialization}.

\index{initialization}

Although floating-point numbers are useful, they are
often a source of confusion because there seems to be an
overlap between integers and floating-point numbers.  For
example, if you have the value {\tt 1}, is that an integer,
a floating-point number, or both?

Strictly speaking, C distinguishes the integer value {\tt 1}
from the floating-point value {\tt 1.0}, even though they
seem to be the same number.  They belong to
different types, and strictly speaking, you are not allowed
to make assignments between types.  For example, the following
is illegal:

\begin{verbatim}
    int x = 1.1;
\end{verbatim}
%
Because the variable on the left is an {\tt int}
and the value on the right is a {\tt double}.  But it is easy
to forget this rule, especially because there are places where C
automatically converts from one type to another (this is called implicit casting, more on that in a bit).
For example,

\begin{verbatim}
    double y = 1;
\end{verbatim}
%
should technically not be legal, but C allows it by converting the
{\tt int} to a {\tt double} automatically.  This is convenient for the programmer,
but it can cause problems; for example:

\begin{verbatim}
    double y = 1 / 3;
\end{verbatim}
%
You might expect the variable {\tt y} to be given the value
{\tt 0.333333}, which is a legal floating-point value, but in
fact it will get the value {\tt 0.0}.  The reason is that the
expression on the right is evaluated as integer division. In C, 
an an integer divided by an integer will be evaluated with integer division.
The result of integer division will always be the whole number of the standard division result.
This means {\tt 5/2 = 2} and {\tt 1/3 = 0}. In the examples above, after the integer division results in {\tt 0}, 
it is converted to floating-point value with the assignment, and it finally 
results in the value {\tt 0.0}.

\index{casting}

One way to solve this problem is to make the right-hand side a floating-point
expression:

\begin{verbatim}
    double y = 1.0 / 3.0;
\end{verbatim}
%
This sets {\tt y} to {\tt 0.333333}, because if either value in the division is a float or double, the compiler will use floating-point division.

\index{arithmetic!floating-point}

All the operations we have seen---addition, subtraction,
multiplication, and division---work on floating-point values,
although you might be interested to know that the underlying mechanism
is completely different.  In fact, most processors have special
hardware just for performing floating-point operations.

\section{Constants}
\label{sec:Constants}
\index{constants}
\index{constant values}

In the previous section we have assigned the value 3.14159 to a
floating point variable. An important thing to remember about variables
is, that they can hold -- as their name implies -- 
different values at different points in your program. 
For example, we could assign
the value 3.14159 to the variable {\tt pi} now and assign 
some other value to it later on:

\begin{verbatim}
    double pi = 3.14159;
    ...
    pi = 10.999;  /* probably a logical error in your program */
\end{verbatim}
%

The second value is probably not what you intended when you first created the named
storage location {\tt pi} . The value for $\pi$ is  constant and does not
change over time. Using the storage location {\tt pi} to hold arbitrary other values can 
cause some very hard to find bugs in your program.

C allows you to specify the static nature of storage locations through
the use of the keyword {\tt const}. It must be used in conjunction with the 
required type of the constant. A value will be assigned at initialization but can
never be changed again during the runtime of the program.  

\begin{verbatim}
    const double PI = 3.14159;
    printf ("Pi: %lf\n", PI);
     ...
    PI = 10.999;  /* wrong, error caught by the compiler  */
\end{verbatim}
%

It is no longer possible to change the value for  {\tt PI} once it has been initialized, but 
other than this we can use it just like a variable.

In order to visually separate constants from variables we will use
all uppercase letters in their names. 


%todo: typecasting �berarbeiten  - siehe deutsche Folien

\section{Converting types}
\label{rounding}
\label{typecasting} 
\index{rounding}
\index{typecasting}

As I mentioned, C converts {\tt int}s
to {\tt double}s automatically if necessary, because no
information is lost in the translation.  On the other hand,
going from a {\tt double} to an {\tt int} requires rounding
off.  C doesn't perform this operation automatically, in
order to make sure that you, as the programmer, are aware
of the loss of the fractional part of the number.

The simplest way to convert a floating-point value to an integer is to
use a {\bf typecast}.  Typecasting is so called because it allows you
to take a value that belongs to one type and ``cast'' it into another
type (in the sense of molding or reforming, not throwing).

The syntax for typecasting requires the explicit specification of
the target type, set in parenthesis before the expression {\tt (Type)}.
For example:

\begin{verbatim}
  const double PI = 3.14159;
  int x = (int) PI;
\end{verbatim}
%
The {\tt (int)} operator casts the value of PI into an integer, so {\tt x} gets the value
3.  Converting to an integer always truncates the double -- it cuts off the fractional part of the value. 
This is essentially  rounding down, even if the fraction part is 0.99999999. 

Of course we can cast an {\tt int} to a {\tt double} without any worry about loss of 
information because we are only adding .0 and not changing the value. 
\begin{verbatim}
	 int x = 3;
	 double y = (double) x;  /* y = 3.0 */
\end{verbatim}
%
We can also directly convert from {\tt char} to {\tt int} because each ASCII character is stored 
as an integer value. If an integer is between 0 and 255 (inclusively) we can also cast to the ASCII character.
\begin{verbatim}
 	char lettter = 'A'
 	int x = 65;
 	char letterX = (char)x;
 	int y = (int) letter;
\end{verbatim}
%
Type can change how C treats values and operations.  
We saw that dividing two {\tt int}s results in integer division. 
However,  when mixing {\tt int} and {\tt double} in arithmetic operations the result will always be a {\tt double}. C will implicitly cast values depending on the operation being performed. Some operations/conversion don't really ``make sense.'' Can you cast a  {\tt double} like 3.12 to a {\tt char}? We can add an {\tt int} to a {\tt char} (because they are really both integers), but can you add a {\tt double} and a {\tt char}?
\begin{verbatim}
 	int x = 3;
 	double y = (double) x;   /* explicit cast */
 	double z = x;  /*implicit cast*/
 	double m = x + 3;  /* implcit cast*/
 	/*m = 6.0 even though x + 3 = 6*/
\end{verbatim}
%

\section{Math functions}
\index{math function}
\index{function!math}
\index{expression}
\index{argument}

In mathematics, you have probably seen functions like $\sin$ and
$\log$, and you have learned to evaluate expressions like
$\sin(\pi/2)$ and $\log(1/x)$.  First, you evaluate the
expression in parentheses, which is called the {\bf argument} of the
function.  For example, $\pi/2$ is approximately 1.571, and $1/x$ is
0.1 (if $x$ happens to be 10).

Then you can evaluate the function itself, either by looking it up in
a table or by performing various computations.  The $\sin$ of 1.571 is
1, and the $\log$ of 0.1 is -1 (assuming that $\log$ indicates the
logarithm base 10).

This process can be applied repeatedly to evaluate more complicated
expressions like $\log(1/\sin(\pi/2))$.  First we evaluate the
argument of the innermost function, then evaluate the function,
and so on.

C provides a set of built-in functions that includes most
of the mathematical operations you can think of.
The math functions are invoked using a syntax that is similar to
mathematical notation:

\begin{verbatim}
     double log = log (17.0);
     double angle = 1.5;
     double height = sin (angle);
\end{verbatim}
%
The first example sets {\tt log} to the logarithm of 17, base
$e$.  There is also a function called {\tt log10} that takes
logarithms base 10.

The second example finds the sine of the value of the variable
{\tt angle}.  C assumes that the
values you use with {\tt sin} and the other trigonometric functions
({\tt cos}, {\tt tan}) are in {\em radians}.  To
convert from degrees to radians, you can divide by 360
and multiply by $2 \pi$.  

If you don't happen to know $\pi$ to 15 digits, you can
calculate it using the {\tt acos} function.  The arccosine
(or inverse cosine) of -1 is $\pi$, because the cosine of
$\pi$ is -1.

\begin{verbatim}
  const double PI = acos(-1.0);
  double degrees = 90;
  double angle = degrees * 2 * PI / 360.0;
\end{verbatim}
%
Before you can use any of the math functions, you have to
include the math {\bf header file}.  You may also need to use the -ml option at compile.
Header files contain information the compiler needs about functions that are defined
outside your program. For example, in the ``Hello, world!''
program we included a header file named {\tt stdio.h} using
an {\bf include} statement:

\begin{verbatim}
      #include <stdio.h>
\end{verbatim}
%
{\tt stdio.h} contains information about input and output
(I/O) functions available in C.

\index{header file}
\index{<stdio.h>}
\index{<math.h>}
\index{header file!stdio.h}
\index{header file!math.h}

Similarly, the math header file contains information
about the math functions.  You can include it at the beginning
of your program along with {\tt stdio.h}:

\begin{verbatim}
    #include <math.h>
\end{verbatim}
%
\section {Composition}
\label{composition}
\index{composition}
\index{expression}

Just as with mathematical functions, C functions can be {\bf
composed}, meaning that you use one expression as part of another.
For example, you can use any expression as an argument to a function:

\begin{verbatim}
    double x = cos (angle + PI/2);
\end{verbatim}
%
This statement takes the value of {\tt PI}, divides it by two and
adds the result to the value of {\tt angle}.  The sum is
then passed as an argument to the {\tt cos} function.

You can also take the result of one function and pass it as
an argument to another:

\begin{verbatim}
    double x = exp (log (10.0));
\end{verbatim}
%
This
statement finds the log base $e$ of 10 and then raises $e$ to that
power.  The result gets assigned to {\tt x}; I hope you know what it
is.


\section{Adding new functions}
\index{function!definition}
\index{main}
\index{function!main}
\index{function!prototype}
So far we have only been using the functions that are built into C,
but it is also possible to add new functions.  Actually, we have already
seen one function definition: {\tt main()}.  The function named {\tt main()}
is special because it indicates where the execution of the program
begins, but the syntax for {\tt main()} is the same as for any other
function definition:

\begin{verbatim}
   void NAME ( LIST OF PARAMETERS ) 
   {
       STATEMENTS
   }
\end{verbatim}
%
You can make up any name you want for your function, except
that you can't call it {\tt main} or any other
C keyword.  The list of
parameters specifies what information, if any, you have to
provide in order to use (or {\bf call}) the new function.

{\tt main()} doesn't take any parameters, as indicated by
the parentheses containing the keyword {\tt (void)} in it's definition.  The first couple
of functions we are going to write also have no parameters, so the
definition looks like this:
\begin{verbatim}
 	 void printNewLine(void) 
 	 {
  	        	printf ("%c", '\n');
 	 }
\end{verbatim}
%
This function is named {\tt printNewLine()}. It contains only a single
statement, which outputs a newline character. Notice that we start
the function name with an lowercase letter. The following words of the
function name are also capitalized. We will use this convention, 
often called camel case because of the humps the capital letters create,  for the naming 
of functions consistently throughout the book. 


A function definition should always have a corresponding {\bf function prototype}. 
These tell the compiler the name of the function, the number and types of arguments it takes, 
and the type of value it returns (if any). Prototypes are written above the main program (or in a header file). 
Function definitions should be written below main. The prototype for the  {\tt printNewLine()} function would be:
\vskip 0.5em
\begin{verbatim}
	 	void printNewLine (void);
\end{verbatim}
%
\vskip 0.5em
Function prototypes aid in error checking by allowing the compiler to check the function calls in your code 
against the function prototypes ensuring that the correct number and types of arguments are being passed.  
This can also help keep code more readable and organized by keeping the main function near the top 
of your code while also providing a sort of quick view list of all the function at the top of the program.


Once we have a prototype and definition, we can call this few function in {\tt main()} using syntax that
is similar to the way we call the built-in C commands:
\vskip 0.5em
\begin{verbatim}
 	void printNewLine (void);  /*function prototype*/

 	 int main (void) {
    	 printf ("First Line.\n");
     	printNewLine ();                  /*function call*/
     	printf ("Second Line.\n");
     	return EXIT_SUCCESS;
  	}
   void printNewLine (void)   /*function definition*/
   {
       printf ("\n");
   }
\end{verbatim}
%
The output of this program is:
\vskip 0.5em

\begin{verbatim}
   First line.

   Second line.
\end{verbatim}
%
Notice the extra space between the two lines.  What if we wanted
more space between the lines?  We could call the same
function repeatedly:

\begin{verbatim}
  int main (void)
  {
      printf ("First Line.\n");
      printNewLine ();
      printNewLine ();
      printNewLine ();
      printf ("Second Line.\n");
  }
\end{verbatim}
%
Or we could write a new function, named {\tt printThreeLines()}, that 
prints three new lines:

\begin{verbatim}
  	void printNewLine (void); 
  	void printThreeLines (void);
  	int main (void)
  	{
    		printf ("First Line.\n");
    		printThreeLines ();
	    	printf ("Second Line.\n");
    		return EXIT_SUCCESS;
 	 }
   void printThreeLines (void)
  	{
	    	printNewLine ();  printNewLine ();  printNewLine (); 
	    	 /*this is legal but maybe better with each on it's own line*/
  	}
   void printNewLine (void)  
   {
  	    printf ("\n");
   }
\end{verbatim}
%
You should notice a few things about this program:

\begin{itemize}

\item You can call the same procedure repeatedly.  In
fact, it is quite common and useful to do so.

\item You can have one function call another function.  In this
case, {\tt main()} calls {\tt printThreeLines()} and {\tt printThreeLines()}
calls {\tt printNewLine()}.  Again, this is common and useful.

\item In {\tt printThreeLines()} I wrote three statements all on the
same line, which is syntactically legal (remember that spaces
and new lines usually don't change the meaning of a program).
On the other hand, it is usually a better idea to put each
statement on a line by itself, to make your program easy to
read.  I sometimes break that rule in this book to save space.

\end{itemize}

So far, it may not be clear why it is worth the trouble to
create all these new functions.  Actually, there are a lot
of reasons, but this example only demonstrates two:

\begin{enumerate}
\item Functions are a form of abstraction and abstraction reduces cognitive load (makes it easier to think)

\item Creating a new function gives you an opportunity to
give a name to a group of statements. Functions can simplify a program
by hiding a complex computation behind a single command, and by using
English words in place of arcane code.  Which is clearer, {\tt
printNewLine()} or {\tt printf("$\backslash$n")}?

\item Creating a new function can make a program smaller by eliminating
repetitive code.  For example, a short way to print nine consecutive
new lines is to call {\tt printThreeLines()} three times.  How would you
print 27 new lines?

\item Functions isolate code into contained areas. This can make it easier to fix issues, make changes, and find bugs

\end{enumerate}

\section {Definitions and uses}

Pulling together all the code fragments from the previous
section, the whole program looks like this:

\begin{verbatim}
  #include <stdio.h>
  #include <stdlib.h>

 void printNewLine (void);
 void printThreeLines (void);
  int main (void)
  {
     printf ("First Line.\n");
     printThreeLines ();
     printf ("Second Line.\n");
     return EXIT_SUCCESS;
  }

	void printNewLine (void) 
	{
	     printf ("\n");
	}

	void printThreeLines (void)
	{
	   printNewLine ();  
	   printNewLine ();  
	   printNewLine ();
	}

\end{verbatim}
%
This program contains three function definitions: {\tt PrintNewLine()},
{\tt printThreeLine()}, and {\tt main()}.

Inside the definition of {\tt main()}, there is a statement that
uses or calls {\tt printThreeLine()}.  Similarly, {\tt printThreeLine()} calls
{\tt printNewLine()} three times.  

Without function prototypes,  the definition of each
function would need to appear above the place it is used -- filling the 
top of our program with definitions and putting main down at the bottom appears above the place where it is used. 
With prototypes, the order of function definitions doesn't matter (so long as the prototypes are at the top of the program)


\section {Programs with multiple functions}

When you look at the C source code remember execution always begins at the first statement of {\tt main()},
regardless of where it is in the program.
Statements are executed one at a time, in order, until you reach a
function call.  Function calls are like a detour in the flow of
execution.  Instead of going to the next statement, you go to the
first line of the called function, execute all the statements there,
and then come back and pick up again where you left off.

That sounds simple enough, except that you have to remember that one
function can call another.  Thus, while we are in the middle of {\tt
main()}, we might have to go off and execute the statements in {\tt
printThreeLines()}.  But while we are executing {\tt printThreeLines()}, we get
interrupted three times to go off and execute {\tt printNewLine()}.

Fortunately, C is adept at keeping track of where it is, so
each time {\tt printNewLine()} completes, the program picks up where it left
off in {\tt printThreeLine()}, and eventually gets back to {\tt main()} so the
program can terminate.

What's the moral of this sordid tale?  When you read a program, don't
read from top to bottom.  Instead, follow the flow of execution.

\section {Parameters and arguments}
\index{parameter}
\index{argument}

Some of the built-in functions we have used have {\bf parameters},
which are values that you provide to let the function do its
job.  For example, if you want to find the sine of a number,
you have to indicate what the number is.  Thus, {\tt sin()}
takes a {\tt double} value as a parameter.

Some functions take more than one parameter, like {\tt pow()},
which takes two {\tt doubles}, the base and the exponent.

Notice that in each of these cases we have to specify not
only how many parameters there are, but also what type they
are.  So it shouldn't surprise you that when you write a
function definition, the parameter list indicates the type of
each parameter.  For example:

\begin{verbatim}
  void printTwice (char phil) 
  {
      printf("%c%c\n", phil, phil);
  }
\end{verbatim}
%
This function takes a single parameter, named {\tt phil}, that
has type {\tt char}.  Whatever that parameter is (and at
this point we have no idea what it is), it gets printed
twice, followed by a newline.
I chose the name {\tt phil} to suggest that the name
you give a parameter is up to you, but in general you want to
choose something more illustrative than {\tt phil}. A better name 
parameter may be something like {\tt charToPrint} or {\tt symbol}

In the function definition the parameter (also called the  ``formal parameter'') has no value, you can think of it like a placeholder.
When we call this function, we have to provide a {\tt char}. This is the argument (or the ``actual parameter'')
and it provides the value that replaces the formal parameter in the definition.

For example, we might have a {\tt main()} function like this:

\begin{verbatim}
  int main (void) 
  {
      printTwice ('a');
      return EXIT_SUCCESS;
  }
\end{verbatim}
%
The {\tt char} value you provide is called an {\bf argument}, and we
say that the argument is {\bf passed} to the function.  In this
case the value {\tt 'a'} is passed as an argument
to {\tt printTwice()} where it will get printed twice.

Alternatively, if we had a {\tt char} variable, we could
use it as an argument instead:

\begin{verbatim}
  int main () 
  {
      char argument = 'b';
      PrintTwice (argument);
      return EXIT_SUCCESS;
  }
\end{verbatim}
%
Notice something very important here: the name of the variable we pass
as an argument ({\tt argument}) has nothing to do with the name of the
parameter ({\tt phil}).  Let me say that again:

\begin{quote}

{\bf The name of the variable we pass as an argument has nothing to do
with the name of the parameter.}

\end{quote}

They can be the same or they can be different, but it is important
to realize that they are not the same thing, except that they happen
to have the same value (in this case the character {\tt 'b'}).

The value you provide as an argument must have the same type as
the parameter of the function you call.  This rule is
important, but it is sometimes confusing because C sometimes
converts arguments from one type to another automatically.  For
now you should learn the general rule, and we will deal with
exceptions later.

\section {Parameters and variables are local}

Parameters and
variables only exist inside their own functions.  Within the
confines of {\tt main()}, there is no such thing as {\tt phil}.
If you try to use it, the compiler will complain.  Similarly,
inside {\tt printTwice()} there is no such thing as {\tt argument}.

Variables like this are said to be {\bf local}.  In order to
keep track of parameters and local variables, it is useful to
draw a {\bf stack diagram}.  Like state diagrams, stack diagrams
show the value of each variable, but the variables are contained
in larger boxes that indicate which function they belong to.

For example, the stack diagram for {\tt printTwice()} looks 
like this:

\vspace{0.1in}
\centerline{\epsfig{figure=figs/stack.pdf,width=6cm}}
\vspace{0.1in}
%
Whenever a function is called, it creates a new {\bf instance}
of that function.  Each instance of a function contains the
parameters and local variables for that function.  In the
diagram an instance of a function is represented by a box
with the name of the function on the outside and the variables
and parameters inside.

In the example, {\tt main()} has one local variable, {\tt argument}, and
no parameters.  {\tt printTwice()} has no local variables and one
parameter, named {\tt phil}.

\section {Functions with multiple parameters}
\index{parameter!multiple}
\index{function!multiple parameter}
%\index{class!Time}

The syntax for declaring and invoking functions with multiple
parameters is a common source of errors.  First, remember
that you have to declare the type of every parameter.  For
example

\begin{verbatim}
  void printTime (int hour, int minute) 
  {
    printf ("%i", hour);
    printf (":");
    printf ("%i", minute);
  }
\end{verbatim}
%
It might be tempting to write {\tt (int hour, minute)}, but
that format is only legal for variable declarations, not
for parameters.

Another common source of confusion is that you do not have
to declare the types of arguments.  The following is wrong!

\begin{verbatim}
    int hour = 11;
    int minute = 59;
    printTime (int hour, int minute);   /* WRONG! */
\end{verbatim}
%
In this case, the compiler can tell the type of {\tt hour}
and {\tt minute} by looking at their declarations.  It is
unnecessary and illegal to include the type when you pass them
as arguments.  The correct
syntax is {\tt printTime (hour, minute);}.

\section {Functions with results}
\index{fruitful function}
\index{function!fruitful}

You might have noticed by now that some of the functions we are using,
like the math functions, yield results.  Other functions,
like {\tt printNewLine}, perform an action but
don't report a result. We say a function returns a value of a specif type or if nothing is returned then we say it has void return.
That raises some questions:

\begin{itemize}

\item What happens if you call a function and you don't
do anything with the result (i.e. you don't assign it to
a variable or use it as part of a larger expression)?

\item What happens if you use a function without a result as part
of an expression, like {\tt printNewLine() + 7}?

\item Can we write functions that yield results, or are we
stuck with things like {\tt printNewLine()} and {\tt printTwice()}?

\end{itemize}

The answer to the third question is ``yes, you can write functions that
return values,'' and we'll do it in a couple of chapters.  I will
leave it up to you to answer the other two questions by trying them
out.  Any time you have a question about what is legal or
illegal in C, a good way to find out is to ask the compiler.

\section{Glossary}

\begin{description}
\item[casting:]  Converting from one type to another. This can be explicit or implicit.

\item[constant:] A named storage location similar to a variable, that can not be changed
once it has been initialized.

\item[floating-point:] A type of variable (or value) that can contain
fractions as well as integers.  There are a few floating-point types
in C; the one we use in this book is {\tt double}.

\item[initialization:]  A statement that declares a new variable
and assigns a value to it at the same time.

\item[function:]  A named sequence of statements that performs some
useful function.  Functions may or may not take parameters, and may
or may not produce a result.

\item[parameter:]  A piece of information you provide
in order to call a function.  Parameters are like variables in
the sense that they contain values and have types.

\item[argument:]  A value that you provide when you call a
function.  This value must have the same type as the corresponding
parameter.

\item[call:]  Cause a function to be executed.

\item[void:] A type that represents no type. It is used to signify a function takes no parameters and/or reports no value

\index{floating-point}
\index{function}
\index{parameter}
\index{argument}
\index{call}
\index{initialization}
\index{void}
\index{casting}
\index{typecasting}


\end{description}


\section{Exercises}
\setcounter{exercisenum}{0}


% LaTeX source for textbook ``How to think like a computer scientist''
% Copyright (C) 1999  Allen B. Downey
% Copyright (C) 2009  Thomas Scheffler
% Copyright (C) 2023  Michael Penta


\begin{exercise}
	
Evaluate each of the following expressions to determine the resulting value. 
Use the following variables and values when you evaluate the expressions:
\begin{verbatim}
    int x = 2;
    double y = 1.2;
    char z = 'A';
\end{verbatim}
%
\begin{enumerate}
	\item {\tt x + 1}
	\item {\tt x + y}
	\item {\tt x / 3}
	\item {\tt x / 3.0}
	\item {\tt (int) y}
	\item {\tt (int) z}
\end{enumerate}
\end{exercise}

%%%%%%%%%%%%%%%%%%%%%%%%%%%%%%%%%
\begin{exercise}

The point of this exercise is to practice reading code and to
make sure that you understand the flow of execution through
a program with multiple functions.

\begin{enumerate}

\item What is the output of the following program?  Be precise
about where there are spaces and where there are newlines.

HINT: Start by describing in words what {\tt ping()} and
{\tt baffle()} do when they are invoked.

\begin{verbatim}

 #include <stdio.h>
 #include <stdlib.h>
 
 void ping(void);
 void baffle(void);
 void zoop(void);
 
    int main(void)
    {
        printf("No, I ");
        zoop();
        printf("I ");
        baffle();
        
        return EXIT_SUCCESS;
    }
    void ping(void)
    {
        printf(".\n"); 
    }
    void baffle(void)
    {
        printf("wug");
        ping(); 
    }
    void zoop(void)
    {
        baffle();
        printf("You wugga ");
        baffle();
    }
    
\end{verbatim}
%

\item Draw a stack diagram that shows the state of the program
the first time {\tt ping()} is invoked.

\end{enumerate}

\end{exercise}

%%%%%%%%%%%%%%%%%%%%%%%%%%%%%%%%%

\begin{exercise}
The point of this exercise is to make sure you understand how
to write and invoke functions that take parameters.

\begin{enumerate}

\item Write a function prototype for a function named {\tt zool()} that
takes three parameters: an {\tt int} and two {\tt char}.

\item Write a line of code that invokes {\tt zool()}, passing
as arguments the value {\tt 11}, the letter {\tt a}, and the letter {\tt z}.
\end{enumerate}


\end{exercise}


%%%%%%%%%%%%%%%%%%%%%%%%%%%%%%%%%%

\begin{exercise}

The purpose of this exercise is to take code from a previous exercise
and encapsulate it in a function that takes parameters.  You should
start with a working solution to exercise
%~\ref{ex.date}.

\begin{enumerate}

\item Write a function definition and prototype for a function called {\tt printDateAmerican()}
that takes the day, month and year as parameters and that
prints them in American format.

\item Test your function by invoking it from {\tt main()} and passing
appropriate arguments.  The output should look something like this
(except that the date might be different):
%
\begin{verbatim}
3/29/2022
\end{verbatim}
%
\item Once you have successfully run the{\tt printDateAmerican()}, write another
function called {\tt printDateEuropean()} that prints the date in
European format.

\item Once you have the two functions working, write a main program that asks the user for the day, month, and year and scan the values into variables.
Call each of your functions with the given input. Do this three times - each time prompt and scan the data from the user and call both functions with the data.

\end{enumerate}

%%%%%%%%%%%%%%%%%%%%%%%%%%%%%%%%%%
\end{exercise}

\begin{exercise}
\label{ex.multadd}

Many computations can be expressed concisely using the ``multadd''
operation, which takes three operands and computes {\tt a * b + c}.  Some
processors even provide a hardware implementation of this operation for
floating-point numbers.

\begin{enumerate}

\item Write a function definition and prototype for a function called {\tt multadd()} that takes three {\tt doubles}
as parameters and that prints their multadditionization.

\item Write a {\tt main()} function that tests {\tt multadd()} by invoking it with a
few simple parameters, like {\tt 1.0, 2.0, 3.0}, and then prints
the result, which should be {\tt 5.0}.

\item Also in {\tt main()}, use {\tt multadd()} to compute the
following value:


{\bf NOTE:} Do not let the math scare you -- you don't have to understand the math to write the code.
Break down each piece. Leverage variables and functions. Look for patterns.

%
\begin{eqnarray*}
& \sin \frac{\pi}{4} + \frac{\cos \frac{\pi}{4}}{2} & 
%\\
%\\
%& \log 10 + \log 20 &
\end{eqnarray*}
%
\item Write a function called {\tt yikes()} that takes a
double as a parameter and that uses {\tt multadd()} to calculate
and print
%
\begin{eqnarray*}
x e^{-x} + \sqrt{1 - e^{-x}}
\end{eqnarray*}
%
HINT: the Math function for raising $e$ to a power is {\tt double exp(double x);}.

\end{enumerate}

In the last part, you get a chance to write a function that invokes
a function you wrote.  Whenever you do that, it is a good idea to
test the first function carefully before you start working on the
second.  Otherwise, you might find yourself debugging two functions
at the same time, which can be very difficult.

One of the purposes of this exercise is to practice pattern-matching:
the ability to recognize a specific problem as an instance of a
general category of problems (think about how these meet the pattern of the multadd function).


\end{exercise}




% LaTeX source for textbook ``How to think like a computer scientist''
% Copyright (C) 1999  Allen B. Downey
% Copyright (C) 2009  Thomas Scheffler
% Copyright (C) 2023  Michael Penta

\chapter{Selection structures and recursion}

\label{condrecursion}


\section{Conditional expressions}
\index{control structures}
\index{conditional expression}
\index{condition}

In order to write useful programs, we almost always need the ability
to check certain conditions and change the behavior of the program
accordingly.  In C we can use {\bf control structures} to change the flow of our program.
Control structures utilize {\bf conditional expressions} to execute statements conditionally.

\index{operator!comparison}
\index{comparison!operator}

Conditional expressions are expressions that yield a true or false value. 
In C, any non-zero expression is true. Zero is false. The following are some example expressions and how they would evaluate in C

\begin{verbatim}

    'd'   /* True */
    1     /* True */
    -1    /* True */
    4.1   /* True */
    1 + 2 /* True */
    0     /* False */
    -1 +1 /* False */
	
\end{verbatim}
%

Most conditional expressions use {\bf comparison operators} 

\begin{verbatim}

	x == y /* x equals y */
	x != y /* x is not equal to y */
	x > y  /* x is greater than y */
	x < y  /* x is less than y */
	x >= y /* x is greater than or equal to y */
	x <= y /* x is less than or equal to y */
	
\end{verbatim}
%
Although these operations are probably familiar to you, the
syntax C uses is a little different from mathematical
symbols like $=$, $\neq$ and $\le$.  A common error is
to use a single {\tt =} instead of a double {\tt ==}.  Remember
that {\tt =} is the assignment operator, and {\tt ==} is
a comparison operator.  Also, there is no such thing as
{\tt =<} or {\tt =>}.

It is generally a good idea to make the two sides of a condition operator be the same
type so its best to compare {\tt int}s to {\tt int}s and {\tt double}s to {\tt double}s. With some implicit or explicit casting
you can also compare {\tt int}s with {\tt char}s and {\tt int}s with {\tt double}s of course you can always type cast if need.  
 Unfortunately, at this
point you can't compare strings at all!  There is
a way to compare strings, but we won't get to it for a couple
of chapters.

{\bf It is also important to note that you should only test floating point values using {\tt >} and {\tt <}.}

Due to the limitations of the floating point representation, these numbers cannot be compared for equality. 
If we need to test these for equality we have to determine if the numbers are close enough to each other. 
To calculate this, we first must decided on a tolerance (like .00001) and then compare  this to the difference of the two numbers.


\section{Selection structures: one-way} 
\index{statement!conditional}
\index{structures!selection}
\index{control structures!selection}
\index{one-way selcection}
\index{selection!one-way}

One category of control structures are the selection structures. The simplest selection structure is the {\tt if} statement:

\begin{verbatim}
    if (x > 0) 
    {
        printf ("x is positive\n");
    }
\end{verbatim}
%

The expression in parentheses must be a conditional expression.
If the expression is true, then the statements in brackets get executed.
If the condition is not true, the statements are not executed. 


This is considered one - way selection -- either you perform the task or you skip over it. 
There is only one choice and you select it or you do not.

\section{The modulus operator}
\index{modulus}
\index{operator!modulus}

The modulus operator works on integers (and integer expressions)
and yields the {\em remainder} when the first operand is divided
by the second.  In C, the modulus operator is a percent sign,
{\tt \%}.  The syntax is exactly the same as for other operators:

\begin{verbatim}
    int quotient = 7 / 3;
    int remainder = 7 % 3;
\end{verbatim}
%
The first operator, integer division, yields 2.  The second
operator yields 1.  Thus, 7 divided by 3 is 2 with 1 left over.

The modulus operator turns out to be surprisingly useful.  For
example, you can check whether one number is divisible by
another: if {\tt x \% y} is zero, then {\tt x} is divisible
by {\tt y}.

Also, you can use the modulus operator to extract the rightmost
digit or digits from a number.  For example, {\tt x \% 10} yields
the rightmost digit of {\tt x} (in base 10).  Similarly
{\tt x \% 100} yields the last two digits.


\section{Random numbers}
\label{Random numbers}
\label{random}
\label{pseudorandom}
\index{random number}
\index{deterministic}
\index{nondeterministic}
Most computer programs do the same thing every time they are executed,
so they are said to be {\bf deterministic}.  Usually, determinism is a
good thing, since we expect the same calculation to yield the same
result.  For some applications, though, we would like the
computer to be unpredictable.  Games are an obvious example.

Making a program truly {\bf nondeterministic} turns out to be not
so easy, but there are ways to make it at least seem
nondeterministic.  One of them is to generate {\bf pseudorandom} numbers and
use them to determine the outcome of the program.
Pseudorandom numbers
are not truly random in the mathematical sense, but 
for our purposes, they will do.

\index{header file!stdlib.h}
\index{<stdlib.h>}

C provides a function called {\tt rand()} that generates
pseudorandom numbers.  It is declared in the
header file {\tt stdlib.h}, which contains a variety of ``standard
library'' functions, hence the name.

The return value from {\tt rand()} is an integer between 0 and {\tt
	RAND\_MAX}, where {\tt RAND\_MAX} is a large number (about 2 billion
on my computer) also defined in the header file.  Each time you call
{\tt rand()} you get a different randomly-generated number.  To see a
sample, run this:

\begin{verbatim}

    int x = rand();
    printf("%i\n", x);
    x = rand();
    printf("%i\n", x);
    x = rand();
    printf("%i\n", x);
    x = rand();
    printf("%i\n", x);
    
\end{verbatim}
%
On my machine I got the following output:

\begin{verbatim}
    1804289383
    846930886
    1681692777
    1714636915
\end{verbatim}
%
You will probably get something similar, but different, on yours.

Of course, we don't always want to work with gigantic integers.
More often we want to generate integers between 0 and some
upper bound.  A simple way to do that is with the modulus
operator.  For example:

\begin{verbatim}
    int x = rand();
    int y = x % range + start;
\end{verbatim}
%
Range here is the number of possible consecutive random values we would like. 
Start is the lowest random value we want.  Since {\tt y} is the remainder when {\tt x} is divided by
{\tt range}, the only possible values for {\tt y}
are between 0 and {\tt range - 1}, including both
end points.   Keep in mind, though, that {\tt y} will never
be equal to {\tt range}. The lower bound here is always 0 so if we add a start value we can shift the range to start at a new lower bound.

For example if we want a random number between 1 and 6, inclusively, our range is 6  [1, 2, 3, 4, 5, 6]. 
\begin{verbatim}
    int range = 6;
    int x = rand();
    int y = x % range;
\end{verbatim}
%
However, this code will generate a number between 0 and 5,  inclusively. To start this sequence at 1, we have to add a start value.

\begin{verbatim}
    int range = 6;
    int start = 1;
    int x = rand();
    int y = x % range + start;
\end{verbatim}
%
It is also frequently useful to generate random floating-point values.
A common way to do that is by dividing by {\tt RAND\_MAX}.  For
example:

\begin{verbatim}
    int x = rand();
    double y = (double) x / RAND_MAX;
\end{verbatim}
%
This code sets {\tt y} to a random value between 0.0 and 1.0,
including both end points.  As an exercise, you might want to
think about how to generate a random floating-point value in
a given range; for example, between 100.0 and 200.0.


\section{Random seeds}
\label{Random seeds}
\index{seed}
\index{random}

If you have run the code in this chapter a few times, you might
have noticed that you are getting the same ``random'' values
every time.  That's not very random!

One of the properties of pseudorandom number generators is that
if they start from the same place they will generate
the same sequence of values.  The starting place is called
a {\bf seed}; by default, C uses
the same seed every time you run the program.

While you are debugging, it is often helpful to
see the same sequence over and over.  That way, when you make
a change to the program you can compare the output before and
after the change.

If you want to choose a different seed for the random number
generator, you can use the {\tt srand()} function.  It takes
a single argument, which is an integer between 0 and {\tt RAND\_MAX}.

For many applications, like games, you want to see a different
random sequence every time the program runs.  A common way to
do that is to use a library function like {\tt time()}
to generate something reasonably unpredictable
and unrepeatable, like the number of seconds since January
1970, and use that number as a seed.  The details
of how to do that depend on your development environment but one example is shown here.

\begin{verbatim}

    #include <stdio.h>
    #include <stdlib.h>
    #include <time.h>
     
    int main (void)
    {
        // Use the number of milliseconds from 1970 to seed rand
        srand(time(0)); 
           	
        // Random number from 1 and 6 
        int x = rand() % 6 + 1;
        printf ("%i", x);
        
        // Random number from -5 to positive 4
        x = rand() % 10 - 5;
        printf ("%i", x);
        
        return EXIT_SUCCESS;
    }
	
\end{verbatim}


Let's look at program that combines selection, modulus, and randomness. The following program generates a random number, either 1 for heads or 2 for tails. The result is printed.

\begin{verbatim}

    #include <stdio.h>
    #include <stdlib.h>
    #include <time.h>
    
    int main (void)
    {
        // Use the number of milliseconds from 1970 to seed rand
        srand(time(0)); 
        
        //random number 1 or 2
        int flip = rand() % 2 + 1;

        if (flip == 1)
        {
            puts("Heads");
        }
        if (flip == 2)
        {
            puts("Tails");
        }
	
 	    return EXIT_SUCCESS;
    }
	
\end{verbatim}



\section {Selection structures: two-way}
\label{alternative}
\index{conditional!alternative}
\index{structures!selection}
\index{control structures!selection}
\index{two-way selcection}
\index{selection!two-way}

A second form of the selection structures is two-way selection, 
in which there are two possibilities, and the condition determines
which one gets executed.  We do one thing or we do another thing, 
unlike one-way selection where we did the thing or we didn't do the thing
 The syntax looks like:

\begin{verbatim}

    if (x % 2 == 0)
    {
        printf("x is even\n");
    } 
    else 
    {
        printf("x is odd\n");
    }
  
\end{verbatim}
%
If the remainder when {\tt x} is divided by 2 is zero, then
we know that {\tt x} is even, and this code displays a message
to that effect.  If the condition is false, the second
set of statements is executed.  Since the condition must
be true or false, exactly one of the alternatives will be
executed.

As an aside, if you think you might want to check the parity
(evenness or oddness) of numbers often, you might want to
``wrap'' this code up in a function, as follows:

\begin{verbatim}

    void printParity(int x) 
    {
        if (x % 2 == 0) 
        {
            printf("x is even\n");
        } 
        else 
        {
            printf("x is odd\n");
        }
    }
  
\end{verbatim}
%
Now you have a function named {\tt printParity()} that will display
an appropriate message for any integer you care to provide.
In {\tt main()} you would call this function as follows:

\begin{verbatim}
   printParity(17);
\end{verbatim}
%
Always remember that when you {\em call} a function, you do
not have to declare the types of the arguments you provide.
C can figure out what type they are based on the definition and the prototype.
You should resist the temptation to write things like:

\begin{verbatim}
   int number = 17;
   printParity(int number); /* WRONG!!! */
\end{verbatim}

\section {Chaining}
\index{chain}
\index{selection!chaining}

Sometimes you want to check for a number of related conditions
and choose one of several actions.  One way to do this is by
{\bf chaining} a series of {\tt if}s and {\tt else}s:

\begin{verbatim}

    if (x > 0) 
    {
        printf ("x is positive\n");
    } 
    else if (x < 0) 
    {
        printf ("x is negative\n");
    } 
    else 
    {
        printf ("x is zero\n");
    }
    
\end{verbatim}
%
These chains can be as long as you want, although they can
be difficult to read if they get out of hand.  One way to
make them easier to read is to use standard indentation,
as demonstrated in these examples.  If you keep all the
statements and squiggly-braces lined up, you are less
likely to make syntax errors and you can find them more
quickly if you do.

\section{Nested conditionals}
\index{conditional!nested}

In addition to chaining, you can also nest one control structures
within another.  We could have written the previous example
as:

\begin{verbatim}

    if (x == 0) 
    {
        printf ("x is zero\n");
    } 
    else 
    {
        if (x > 0) 
        {
            printf ("x is positive\n");
        }
        else 
        {
            printf ("x is negative\n");
        }
    }
    
\end{verbatim}
%
There is now an outer conditional that contains two branches.  The
first branch contains a simple output statement, but the second
branch contains another {\tt if} statement, which has two branches
of its own.  Fortunately, those two branches are both output
statements, although they could have been conditional statements as
well.

Notice again that indentation helps make the structure
apparent, but nevertheless, nested conditionals get difficult to read
very quickly.  In general, it is a good idea to avoid them when you
can.

\index{nested structure}

On the other hand, this kind of {\bf nested structure} is common, and
we will see it again, so you better get used to it.


\section{Selection structures: switch}
\index{switch}
\index{selection!switch}
\index{control structure!switch}

A switch is sort of short hand to replace some long chained structures. 
You can use a switch when you are testing a single {\tt int} or {\tt char}, you
are testing for equality, and you would otherwise have a long chain. For example we can take this chained structure that is testing the 
int value and printing a message based on the value.

\begin{verbatim}
	
    // x is an int
	
    if (x == 1) 
    {
        puts("Message 1");
    } 
    else if (x == 2) 
    {
        puts("Message 2");
    } 
    else if (x == 3) 
    {
        puts("Message 3");
    } 
    else 
    {
        puts("Default message");
    }
    
\end{verbatim}
%

We can rewrite this as a switch because we are testing an int value for equality in a chained structure. Each block of code in the chained structure 
becomes a case in the switch. Each case in the switch must have a label and a break. Most labels start with {\tt case|} and then have the value we are testing for equality --
like {\tt case 2:} or {\tt case 'c':}. One case has a unique label -- {\tt default}. The default case has no value to test because it is the catchall -- it only runs if all other cases fail. All cases end with the word {\tt break}. A break forces control to leave the switch structure. Breaks are an important part of the switch structure.
Write a program that prompts the user for an integer. Use the switch to print various messages. Try removing some of the breaks and  check out how the changes behave.
Omitting breaks between cases can cause fall through - this can be a bug or a feature depending on how you use it (look into stacking cases in switches).


\begin{verbatim}
	
    //x is an int
    
    switch (x)
    {
        case 1:
           puts("message1");
           break;
           
        case 2:
           puts("message2");
           break;
           
        case 3:
           puts("message3");
           break;
           
        default:
            puts("message4");
            break;
    }
		
\end{verbatim}
%

\section{The {\tt return} statement and early termination}
\index{return}
\index{statement!return}

The {\tt return} statement allows you to terminate the execution
of a function. You can place a return statement in any part of the function. Once the program hits the return it will leave the function and go back to the caller. 
If you put a return statement before you before you reach the end of a function this is called an early return or early termination. This can be useful to guard the function from doing unnecessary action. For example, if the parameter of the function must greater than zero, then we can put an if statement to act as a gaurd and stop the 
execution of the function right off the top.

\begin{verbatim}

    #include <math.h> 
    
    void printLogarithm(double x) 
    {
        if (x > 0.0) /* 'Guard' for early return if x >0 */
        {
            printf("Positive numbers only, please.\n");
            return;  
        }
        
        double result = log(x);
        printf("The log of x is %f\n", result);
    }
  
\end{verbatim}
%
This defines a function named {\tt printLogarithm()} that takes
a {\tt double} named {\tt x} as a parameter.  The first thing
it does is check whether {\tt x} is greater than
zero, in which case it displays an error message and then uses
{\tt return} to exit the function.  The flow of execution
immediately returns to the caller and the remaining lines of
the function are not executed.

Remember that any time you want to use one a function from the math
library, you have to include the header file {\tt math.h} and 
you may be required to link the math library at compile time with -lm (dash L M)

\section{Recursion}
\label{recursion}
\index{recursion}

I mentioned in the last chapter that it is legal for one function to
call another, and we have seen several examples of that.  I neglected
to mention that it is also legal for a function to call itself.  It
may not be obvious why that is a good thing, but it turns out to be
one of the most magical and interesting things a program can do.

For example, look at the following function:

\begin{verbatim}

    void countdown(int n) 
    {
        if (n == 0) 
        {
            printf("Blastoff!");
        }
        else
        {
            printf("%i", n);
            countdown(n - 1);
        }
    }
  
\end{verbatim}
%
The name of the function is {\tt countdown()} and it takes a single
integer as a parameter.  If the parameter is zero, it outputs
the word ``Blastoff.''  Otherwise, it outputs the parameter and
then calls a function named {\tt countdown()}---itself---passing
{\tt n-1} as an argument.

What happens if we call this function like this:

\begin{verbatim}

  int main(void)
  {
       countdown(3);
       
       return EXIT_SUCCESS;
  }
  
\end{verbatim}
%
The execution of {\tt countdown()} begins with {\tt n=3}, and
since {\tt n} is not zero, it outputs the value 3, and then
calls itself...

\begin{quote}
The execution of {\tt countdown()} begins with {\tt n=2}, and
since {\tt n} is not zero, it outputs the value 2, and then
calls itself...

\begin{quote}
The execution of {\tt countdown()} begins with {\tt n=1}, and
since {\tt n} is not zero, it outputs the value 1, and then
calls itself...

\begin{quote}
The execution of {\tt countdown()} begins with {\tt n=0}, and
since {\tt n} is zero, it outputs the word ``Blastoff!''
and then returns.
\end{quote}

The countdown that got {\tt n=1} returns.

\end{quote}

The countdown that got {\tt n=2} returns.

\end{quote}

The countdown that got {\tt n=3} returns.

\noindent And then you're back in {\tt main()} (what a trip).  So the
total output looks like:

\begin{verbatim}
    3
    2
    1
    Blastoff!
\end{verbatim}
%
As a second example, let's look again at the functions
{\tt printNewLine()} and {\tt printThreeLines()}.

\begin{verbatim}

    void printNewLine () 
    {
        printf ("\n");
    }

    void printThreeLines () 
    {
        printNewLine();  printNewLine();  printNewLine();
    }
    
\end{verbatim}
%
Although these work, they would not be much help if I wanted
to output 2 newlines, or 106.  A better alternative would be

\begin{verbatim}

    void printLines(int n) 
    {
        if (n > 0) 
        {
            printf("\n");
            printLines(n - 1);
        }
    }
    
\end{verbatim}
%
This program is similar to {\tt countdown}; as long as {\tt n} is
greater than zero, it outputs one newline, and then calls itself to
output {\tt n-1} additional newlines.  Thus, the total number of
newlines is {\tt 1 + (n-1)}, which usually comes out to roughly {\tt
n}.

\index{recursive}
\index{newline}

The process of a function calling itself is called {\bf recursion}, and
such functions are said to be {\bf recursive}.

\section {Infinite recursion}

In the examples in the previous section, notice that each time the
functions get called recursively, the argument gets smaller by one, so
eventually it gets to zero.  When the argument is zero, the function
returns immediately, {\em without making any recursive calls}.
This case---when the function completes without making a recursive
call---is called the {\bf base case}.

If a recursion never reaches a base case, it will go on making recursive
calls forever and the program will never terminate.  This is known as
{\bf infinite recursion}, and it is generally not considered a good
idea.

\index{recursion!infinite}
\index{infinite recursion}
\index{run-time error}

In most programming environments, a program with an infinite
recursion will not really run forever.  Eventually, something
will break and the program will report an error.  This is the
first example we have seen of a run-time error (an error that
does not appear until you run the program).

You should write a small program that recurses forever and run
it to see what happens.

\section {Tips on writing recursion solutions}
It is helpful to have a strategy to tackle recursive solutions. 
\begin{enumerate}
	\item Identify a base case. This is what stops the recursion. This block will not have a recursive call
	\item Identify the general case (the recursive step). This is the repeating part. It will contain a recursive call
	\item The recursive call will use an argument that gets the next step closer to the base case.
	\item An if statement will test the parameter and determine if the base case is run or the general case.
\end{enumerate}

\section {Stack diagrams for recursive functions}
\index{stack}
\index{stack diagram}
\index{diagram!state}
\index{diagram!stack}

In the previous chapter we used a stack diagram to represent the
state of a program during a function call.  The same kind
of diagram can make it easier to interpret a recursive function.

Remember that every time a function gets called it creates
a new instance that contains
the function's local variables and parameters.

This figure shows a stack diagram for Countdown, called
with {\tt n = 3}:

\vspace{0.1in}
\centerline{\epsfig{figure=figs/stack2.pdf,width=6cm}}
\vspace{0.1in}
%
There is one instance of {\tt main()} and four instances of
{\tt Countdown()}, each with a different value for the parameter
{\tt n}.  The bottom of the stack, {\tt Countdown()} with {\tt n=0}
is the base case.  It does not make a recursive call, so there
are no more instances of {\tt Countdown()}.

The instance of {\tt main()} is empty because {\tt main()} does not
have any parameters or local variables.  As an exercise, draw a
stack diagram for {\tt PrintLines()}, invoked with the parameter {\tt n=4}.


\section{Glossary}

\begin{description}

\item[modulus:]  An operator that works on integers and yields
the remainder when one number is divided by another.  In C
it is denoted with a percent sign ({\tt \%}).

\item[deterministic:]  A program that does the same thing every
time it is run.

\item[pseudorandom:]  A sequence of numbers that appear to be
random, but which are actually the product of a deterministic
computation.

\item[seed:]  A value used to initialize a random number sequence.
Using the same seed should yield the same sequence of values.

\item[condition:]  An expression that results in true or false


\item[control structure:]  A structure that uses a condition to allow for conditionally executing a block of code. 

\item[selection structure:]  A type of control structure. Selection structures include {\tt if}, {\tt if...else}, and {\tt switch}

\item[chaining:]  A way of joining several conditional statements
in sequence.

\item[nesting:] Putting a conditional statement inside one or both
branches of another conditional statement.

\item[recursion:]  The process of calling the same function you
are currently executing.

\item[infinite recursion:]  A function that calls itself
recursively without every reaching the base case.  Eventually
an infinite recursion will cause a run-time error.

\index{modulus}
\index{conditional}

\index{conditional!chained}
\index{conditional!nested}
\index{recursion}
\index{recursion!infinite}
\index{infinite recursion}
\index{random}
\index{random!seed}
\end{description}


\section{Exercises}
\setcounter{exercisenum}{0}


% LaTeX source for textbook ``How to think like a computer scientist''
% Copyright (C) 1999  Allen B. Downey
% Copyright (C) 2009  Thomas Scheffler

\begin{exercise}
Use the variables initialized below to evaluate each expression. State if C would consider the resulting value as true or false.
\begin{verbatim}
    int m = 0;
    int n = 1;
    char p = 'k';
    char s = 'F';
    double t = 1.2;   
\end{verbatim}
\begin{enumerate}
	\item {\tt m / n}
	\item {\tt m + n}
	\item {\tt m}
	\item {\tt p <= s}
	\item {\tt t < n}
\end{enumerate}
\end{exercise}

\begin{exercise}
This exercise reviews the flow of execution through a program
with multiple methods.  Read the following code and answer the
questions below.

\begin{verbatim}

    #include <stdio.h>
    #include <stdlib.h>
    
    void zippo (int, int);
    void baffle (int);
    
    int main (void)
    {
        zippo(5, 13);
        
        return EXIT_SUCCESS;
    }
    
    void baffle (int output)
    {
        printf("%i\n", output);
        zippo(12, -5);
    }
    
    void zippo (int quince, int flag)
    {
        if (flag < 0)
        {
            printf("%i zoop\n", quince);
        }
        else
        {
            printf("rattle ");
            baffle(quince);
            printf("boo-wa-ha-ha\n");
        }
    }
    
\end{verbatim}
%
\begin{enumerate}

\item Write the number {\tt 1} next to the first {\em statement}
of this program that will be executed.  Be careful to distinguish
things that are statements from things that are not.

\item Write the number {\tt 2} next to the second statement, and so on
until the end of the program.  If a statement is executed more than
once, it might end up with more than one number next to it.

\item What is the value of the parameter {\tt quince} when {\tt baffle()}
gets invoked for the first time?

\item What is the exact output of this program? Pay close attention to the printed white space like spaces, tabs, and new lines.

\end{enumerate}

\end{exercise}

\begin{exercise}
	In this exercise you will practice using random with functions
	
		\begin{enumerate}
		\item Define a function and a prototype called rollDie that has no parameters. 
	In the function call rand to generate a random number 1, 2, 3, 4, 5, or 6. Print the resulting value.
	
	\item Define a main function that seeds the random with epoch time (use time() from time.h) and calls your function 3 times
		\end{enumerate}
\end{exercise}

\begin{exercise}
	In this exercise you will practice using selection statements with functions
	\begin{enumerate}
		\item Define a function and a prototype called validate that has one parameter, an int. 
		If the int is 0 the function should print an error message and terminate. 
		If the parameter is non-zero, multiply the value by 2 and print the result.
		
		\item 
		Define a main function that calls your function 3 times with the following arguments: 0, 3, -2
		
	\end{enumerate}

\end{exercise}


\begin{exercise}
There is an old song about 
beer bottles that can be expressed recursively.

The first verse of the song ``99 Bottles of Beer'' is:

\begin{quote}
99 bottles of beer on the wall,
99 bottles of beer,
ya' take one down, ya' pass it around,
98 bottles of beer on the wall.
\end{quote}

Subsequent verses are identical except that the number
of bottles gets smaller by one in each verse, until the
last verse:

\begin{quote}
No bottles of beer on the wall,
no bottles of beer,
ya' can't take one down, ya' can't pass it around,
'cause there are no more bottles of beer on the wall!
\end{quote}
%
And then the song (finally) ends.

Write a program that prints the entire lyrics of
``99 Bottles of Beer.''  Your program should include a
recursive method that does the hard part, but you also
might want to write additional methods to separate the major
functions of the program.

The last verse, when the number of bottles left is 0, is the base case. 
The other verses are the recursive step. 

As you are developing your code, you will probably
want to test it with a small number of verses, like
``3 Bottles of Beer.''

The purpose of this exercise is to take a problem and break it
into smaller problems, and to solve the smaller problems by writing
simple, easily-debugged methods.
\end{exercise}


\begin{exercise}
You can use the  {\tt getchar()} function in C to
get character input from the user through the keyboard.
This function stops the execution of the program and waits
for the input from the user. 

The {\tt getchar()} function has the type {\tt int} and does
not require an argument. It returns the ASCII-Code (cf. Appendix~\ref{ASCII-Table})
of the key that has been pressed on the keyboard.
\begin{enumerate}
\item Write a program, that asks the user 
to input a digit between  0 and 9. 

\item Test the input from the user and display an error message 
if the returned value is not a digit. The program should then
be terminated.
If the test is successful, the program should print the 
input value on the computer screen.
\end{enumerate}


\end{exercise}

% Kapitel 5 (Return)
%Schreiben Sie dazu eine Funktion {\tt AsciiToNumber()} welche ein
%{\tt int} als Typ und als Argument besitzt. �bergeben Sie der Funktion
%den eingelesenen Wert und 




% String--Ausgabe!

%\begin{exercise}
%Lesen Sie das folgende Programm. Notieren Sie die Ausgabe die w�hrend der
%Abarbeitung des Programms erzeugt wird.

%\begin{verbatim}
%  void Zoop (String fred, int bob) 
%  {
%        printf ("%s\n", fred);
%        if (bob == 5) 
%        {
%            ping ("not ");
%        } 
%        else 
%        {
%            printf ("!\n");
%        }
%    }

%  int main (void) 
%  {
%        int bizz = 5;
%        int buzz = 2;
%        zoop ("just for", bizz);
%        clink (2*buzz);
%    }

%  void clink (int fork) 
%  {
%        printf ("It's ");
%        zoop ("breakfast ", fork) ;
%  }

%  void ping (String strangStrung) 
%  {
%        printf ("any %smore \n", strangStrung);
%    }
%}
%\end{verbatim}
%\end{exercise}






% LaTeX source for textbook ``How to think like a computer scientist''
% Copyright (C) 1999  Allen B. Downey
% Copyright (C) 2009  Thomas Scheffler
% Copyright (C) 2023  Michael Penta

\chapter{Fruitful functions}

\section{Return values}
\index{return}
\index{statement!return}
\index{function!fruitful}
\index{fruitful function}
\index{return value}
\index{void}
\index{function!void}

Some of the built-in functions we have used, like the math
functions, have produced results.  That is, the effect of
calling the function is to generate a new value, which we
usually assign to a variable or use as part of an expression.
For example:

\index{math function!exp()}
\index{math function!sin()}

\begin{verbatim}
    double e = exp(1.0);
    double height = radius * sin(angle);
\end{verbatim}
%
But so far all the functions we have written have been {\bf void}
functions; that is, functions that return no value.  When you call
a void function, it is typically on a line by itself, with
no assignment -- there is nothing to store as no result was produced:

\begin{verbatim}
   printLines(3);
   countdown(n - 1);
\end{verbatim}
%
In this chapter, we will create functions that produce results, or ``fruit,'' 
as opposed to our previous void functions, which produced nothing. 
I will refer to as {\bf fruitful} functions because they yield results.

The first example is {\tt area}, which takes a {\tt
double} as a parameter, and returns the area of a circle with the
given radius:

\index{math function!acos()}
\index{pi}

\begin{verbatim}

    double area (double radius) 
    {
        double pi = acos(-1.0); 
        double area = pi * radius * radius;
        
        return area;
    }
    
\end{verbatim}
%
The first thing you should notice is that the beginning of the
function definition is different.  Instead of {\tt void}, which
indicates a void function (that will not produce a fruit), we see {\tt double}, which indicates that
the return value (the fruit) from this function will have type {\tt double}. 

Also, notice that the last line is an alternate form of the
{\tt return} statement that includes a return value.  This
statement means, ``return immediately from this function and
use the following expression as a return value.'' The type of the expression in
the {\tt return} statement must match the return type of the function.
In other words, when you declare that the return type is {\tt double},
you are making a promise that this function will eventually
produce a {\tt double}.  If you try to {\tt return} with no
expression, or an expression with the wrong type, the compiler will
take you to task. The purpose of our function prototypes is to
let the compiler know the type of the parameters and return value of our function.
 
The prototype for this function would be:

\begin{verbatim}
	double area(double);
\end{verbatim}
%

When we define a fruitful function we can only return one value. The return
expression you provide can be arbitrarily complicated, but it must yield only one value. 
We could have written this function more concisely, but ultimately we only return one value:

\begin{verbatim}

    double area(double radius) 
    {
    	return acos(-1.0) * radius * radius;
    }
    
\end{verbatim}
%

\index{temporary variable}
\index{variable!temporary}
\index{variable!local}
\index{prototype!with return}
On the other hand, {\bf temporary}, or {\bf local}, variables like {\tt area} and {\tt pi} often
make debugging easier and help to break up concepts into smaller more manageable parts.

There are two main schools of thought when it comes to returning from functions.

One idea is that functions should have only one return statement, a single exit point. 
Others favor multiple returns. This book takes no stance on this issue -- other than we 
strive to write readable and maintainable code. Sometimes this means we have single exit point, 
while other times using multiple return statements to return early from a function.
We will note that single exit functions can be easier to debug because they
uses local variables to store return results.
We use multiple return statements in this absolute value example:

\begin{verbatim}

    double absoluteValue(double x) 
    {
        if (x < 0) 
        {
            return -x;
        } 
        else 
        {
            return x;
        }
    }
  
\end{verbatim}
%
Since these returns statements are in an alternative conditional, only
one will be executed.  Having more than one
{\tt return} statement in a function, means as soon
as a return is executed, the function terminates without executing any
subsequent statements. This can be used to exit a function when 
we know there is no point in executing the remaining code:

\begin{verbatim}

	/*
	This function returns 0 if either 
	x or y are negative otherwise it returns
	their product
	*/
    double earlyReturnExample(double x) 
    {
        if (x < 0) // Guard from x being negative
        {
            return 0;
        } 

        if (y < 0) // Guard from y being negative
        {
            return 0;
        } 

	return x * y;
}

\end{verbatim}
%

Code that appears after a {\tt return} statement, or any place else
where it can never be executed, is called {\bf dead code}.  Some
compilers warn you if part of your code is dead.

\index{dead code}

If you put return statements inside a conditional, then
you have to guarantee that {\em every possible path} through
the program hits a return statement.  For example:

\begin{verbatim}
    double AbsoluteValue(double x) 
    {
        if (x < 0) 
        {
            return -x;
        } 
        else if (x > 0) 
        {
            return x;
        }
        /* WRONG!! */
    }
\end{verbatim}
%
This program is not correct because if {\tt x} happens to be 0, then
neither condition will be true and the function will end without hitting
a return statement.  Unfortunately, many C compilers do not catch
this error.  As a result, the program may compile and run, but the
return value when {\tt x==0} could be anything, and will probably
be different in different environments.

\index{absolute value}
\index{error!compile-time}
\index{compile-time error}

By now you are probably sick of seeing compiler errors, but as you
gain more experience, you will realize that the only thing worse
than getting a compiler error is {\em not} getting a compiler error
when your program is wrong.

Here's the kind of thing that's likely to happen: you test {\tt
absoluteValue()} with several values of {\tt x} and it seems to work
correctly.  Then you give your program to someone else and they run it
in another environment.  It fails in some mysterious way, and it
takes days of debugging to discover that the problem is an
incorrect implementation of {\tt absoluteValue()}.  If only the
compiler had warned you!

\index{compile-time error}
\index{error!compile-time}
\index{debugging}

From now on, if the compiler points out an error in your program, you
should not blame the compiler.  Rather, you should thank the compiler
for finding your error and sparing you days of debugging.  Some
compilers have an option that tells them to be extra strict and report
all the errors they can find.  You should turn this option on all the
time.

\index{math function!fabs()}

As an aside, you should know that there is a function in the
math library called {\tt fabs()} that calculates the absolute
value of a {\tt double} -- correctly.

\section{Program development}
\label{distance}
\index{program development}

At this point you should be able to look at complete C functions
and tell what they do.  But it may not be clear yet how to go
about writing them.  I am going to suggest one technique that
I call {\bf incremental development}.

\index{incremental development}
\index{program development}

As an example, imagine you want to find the distance between two
points, given by the coordinates $(x_1, y_1)$ and $(x_2, y_2)$.  By
the usual definition,

\begin{equation}
distance = \sqrt{(x_2 - x_1)^2 + (y_2 - y_1)^2}
\end{equation}
%
The first step is to consider what a {\tt Distance} function
should look like in C.  In other words, what are the inputs
(parameters) and what is the output (return value).

In this case, the two points are the parameters, and it is natural to
represent them using four {\tt double}s.  The return value is the
distance, which will have type {\tt double}.

Already we can write an outline of the function:

\begin{verbatim}

    double distance(double x1, double y1, double x2, double y2) 
    {
        return 0.0;
    }
    
\end{verbatim}
%
The {\tt return} statement is a placekeeper so that the function will
compile and return something, even though it is not the right answer.
At this stage the function doesn't do anything useful, but it is
worthwhile to try compiling it so we can identify any syntax errors
before we make it more complicated.

In order to test the new function, we have to call it with
sample values.  Somewhere in {\tt main()} I would add:

\begin{verbatim}
    double dist = distance(1.0, 2.0, 4.0, 6.0);
    printf("%f\n" dist);
\end{verbatim}
%
I chose these values so that the horizontal
distance is 3 and the vertical distance is 4; that way,
the result will be 5 (the hypotenuse of a 3-4-5 triangle).
When you are testing a function, it is useful to know the right
answer. Before you test any code you should 
state what you expect the output to be.

Once we have checked the syntax of the function definition, we
can start adding lines of code one at a time.  After each
incremental change, we recompile and run the program.  That
way, at any point we know exactly where the error must be---in
the last line we added.

The next step in the computation is to find the differences
$x_2 - x_1$ and $y_2 - y_1$.  I will store those values in
temporary variables named {\tt dx} and {\tt dy}.

\begin{verbatim}

    double distance(double x1, double y1, double x2, double y2) 
    {
        double dx = x2 - x1;
        double dy = y2 - y1;
        printf("dx is %f\n", dx);
        printf("dy is %f\n", dy;
        
        return 0.0;
    }
    
\end{verbatim}
%
I added output statements that will let me check the intermediate
values before proceeding.  As I mentioned, I already know that they
should be 3.0 and 4.0.

\index{scaffolding}

When the function is finished I will remove the output statements.  Code
like that is called {\bf scaffolding}, because it is helpful for
building the program, but it is not part of the final product.
Sometimes it is a good idea to keep the scaffolding around, but
comment it out, just in case you need it later.

The next step in the development is to square {\tt dx} and {\tt dy}.
We could use the {\tt pow()} function, but it is simpler and
faster to just multiply each term by itself.

\begin{verbatim}

    double Distance(double x1, double y1, double x2, double y2)
    {
        double dx = x2 - x1;
        double dy = y2 - y1;
        double dsquared = dx * dx + dy * dy;
        printf("dsquared is %f\n", dsquared);
        
        return 0.0;
    }
    
\end{verbatim}
%
Again, I would compile and run the program at this stage
and check the intermediate value (which should be 25.0).

Finally, we can use the {\tt sqrt()} function to compute and
return the result.

\begin{verbatim}

    double distance(double x1, double y1, double x2, double y2) 
    {
        double dx = x2 - x1;
        double dy = y2 - y1;
        double dsquared = dx * dx + dy * dy;
        double result = sqrt(dsquared);
        
        return result;
    }
    
\end{verbatim}
%
Then in {\tt main()}, we should output and check the value of the result.

As you gain more experience programming, you might find yourself
writing and debugging more than one line at a time.  Nevertheless,
this incremental development process can save you a lot of
debugging time.

The key aspects of the process are:

\begin{itemize}

\item Start with a working program and make small, incremental
changes.  At any point, if there is an error, you will know
exactly where it is.

\item Use temporary variables to hold intermediate values so
you can output and check them.

\item Once the program is working, you might want to remove
some of the scaffolding or consolidate multiple statements into
compound expressions, but only if it does not make the program
difficult to read. We call this refactoring the code.

\end{itemize}

\section{Composition}
\index{composition}

As you should expect by now, once you define a new function,
you can use it as part of an expression, and you can build
new functions using existing functions.  For example, what if someone
gave you two points, the center of the circle and a point on
the perimeter, and asked for the area of the circle?

Let's say the center point is stored in the variables {\tt xc}
and {\tt yc}, and the perimeter point is in {\tt xp} and
{\tt yp}.  The first step is to find the radius of the circle, which
is the distance between the two points.  Fortunately, we have
a function, {\tt Distance()}, that does that.

\begin{verbatim}
  double radius = distance(xc, yc, xp, yp);
\end{verbatim}
%
The second step is to find the area of a circle with that
radius, and return it.

\begin{verbatim}
  double result = area(radius);
  return result;
\end{verbatim}
%
Wrapping that all up in a function, we get:

\begin{verbatim}

    double areaFromPoints(double xc, double yc, double xp, double yp) 
    {
        double radius = distance(xc, yc, xp, yp);
        double result = area(radius);
        
        return result;
    } 
    
\end{verbatim}
%


The temporary variables {\tt radius} and {\tt area} are
useful for development and debugging, but once the program is
working we can make it more concise by composing
the function calls, but we should always favor readability over conciseness:

\begin{verbatim}

    double areaFromPoints(double xc, double yc, double xp, double yp) 
    {
        return area(distance(xc, yc, xp, yp));
    }
     
\end{verbatim}


\section{Boolean values}
\index{boolean}
\index{value!boolean}

The types we have seen so far can hold very large values.  There are a lot
of integers in the world, and even more floating-point numbers.
By comparison, the set of characters is pretty small.  Well, many computing
languages implement an even more fundamental type that is even smaller.  It is called
{\bf \_Bool}, and the only values in it are {\tt true} and {\tt false}.

Unfortunately, earlier versions of the C standard did not implement boolean as
a separate type, but instead used the integer values 0 and 1 to represent 
truth values. By convention 0 represents {\tt false} and 1 represents {\tt true}. 
Strictly speaking C interprets any integer value different from 0 as true. This
can be a source of error if you are testing a value to be true by comparing it with {\tt 1}.

%

Without thinking about it, we have been using boolean values in the
last of chapter.  The condition inside an {\tt if}
statement is a boolean expression.
Also, the result of a comparison operator is a boolean value.
For example:

\begin{verbatim}

    if (x == 5) 
    {
        /* Do something */
    }
    
\end{verbatim}
%
The operator {\tt ==} compares two integers and produces a
boolean value.

\index{operator!comparison}
\index{comparison operator}

Pre C99 has no keywords for the expression of {\tt true} or {\tt false}.
A lot of programs instead are
using C preprocessor definitions anywhere a boolean expression is called for.
For example, 

\begin{verbatim}

    #define FALSE 0
    #define TRUE 1
     ...
    if (TRUE) 
    {
        /* Will be always executed  */
    }
    
\end{verbatim}
%
%todo: Loop?
is a standard idiom for a loop that should run forever (or
until it reaches a {\tt return} or {\tt break} statement).

\section{Boolean variables}
\index{type!{\tt bool}}

Since boolean values are not supported directly in C, we can not declare
variables of the type boolean. 
Instead, programmers typically use the {\tt short} datatype in combination with 
preprocessor definitions to store truth values.

\begin{verbatim}

    #define FALSE 0
    #define TRUE 1
     ...
    short fred;
    fred = TRUE;
    short testResult = FALSE;
    
\end{verbatim}
%
The first line is a simple variable declaration;
the second line is an assignment, and the third line is a
combination of a declaration and as assignment, 
called an initialization.

\index{initialization}
\index{statement!initialization}

As I mentioned, the result of a comparison operator is a boolean,
so you can store it in a variable

\begin{verbatim}
    short evenFlag = (n % 2 == 0);     /* true if n is even */
    short positiveFlag = (x > 0);    /* true if x is positive */
\end{verbatim}
%
and then use it as part of a conditional statement later

\begin{verbatim}
    if (evenFlag) 
    {
        printf("n was even when I checked it\n");
    }
\end{verbatim}
%
A variable used in this way is called a {\bf flag},
since it flags the presence or absence of some condition.

\index{flag}

\section{Logical operators}
\index{logical operator}
\index{operator!logical}

There are three {\bf logical operators} in C: AND, OR and NOT,
which are denoted by the symbols {\tt \&\&}, {\tt ||} and
{\tt !}.  The semantics (meaning) of these operators is similar
to their meaning in English.  For example {\tt x > 0 \&\& x < 10}
is true only if {\tt x} is greater than zero AND less than 10.

\index{semantics}

{\tt evenFlag || n\%3 == 0} is true if {\em either} of
the conditions is true, that is, if {\tt evenFlag} is true OR the
number is divisible by 3.

Finally, the NOT operator has the effect of negating or
inverting a bool expression, so {\tt !evenFlag} is true
if {\tt evenFlag} is false; that is, if the number is odd.

\index{nested structure}

Logical operators often provide a way to simplify nested
conditional statements.  For example, how would you write
the following code using a single conditional?

\begin{verbatim}

    if (x > 0) 
    {
        if (x < 10) 
        {
            printf("x is a positive single digit.\n");
        }
    }
    
\end{verbatim}

\section{Bool functions}
\label{bool}
\index{bool}
\index{function!bool}

It is sometimes appropriate for functions to return {\tt boolean} values just 
like any other return type. This is 
is especially convenient for hiding complicated tests inside
functions.  For example:

\begin{verbatim}

    int IsSingleDigit(int x)
    {
        if (x >= 0 && x < 10) 
        {
            return TRUE;
        } 
        else 
        {
            return FALSE;
        }
    }
    
\end{verbatim}
%
The name of this function is {\tt isSingleDigit()}.  It is common
to give such test functions names that sound like yes/no questions.
The return type is {\tt int}, which means that again we need
to follow the agreement that  0 represents {\tt false} and 1 
represents {\tt true}. Every return
statement has to follow this convention, again, we are using
preprocessor definitions.

The code itself is straightforward, although it is a bit longer than
it needs to be.  Remember that the expression {\tt x >= 0 \&\& x < 10}
is evaluated to a {\tt boolean} value, so there is nothing wrong with returning it
directly, and avoiding the {\tt if} statement altogether:

\begin{verbatim}

    int iSingleDigit(int x)
    {
        return (x >= 0 && x < 10);
    }
    
\end{verbatim}
%
In {\tt main()} you can call this function in the usual ways:

\begin{verbatim}
    printf("%i\n", isSingleDigit(2));
    short bigFlag = !isSingleDigit(17);
\end{verbatim}
%
The first line outputs the value {\tt true} because 2 is a
single-digit number.  Unfortunately, when C outputs {\tt boolean} values, it
does not display the words {\tt TRUE} and {\tt FALSE}, but rather the
integers {\tt 1} and {\tt 0}.
%\footnote{There is a way to fix that
%using the {\tt boolalpha} flag, but it is too hideous to mention.}

The second line assigns the value {\tt true} to {\tt bigFlag}
only if 17 is {\em not} a positive single-digit number.

The most common use of boolean functions is inside conditional
statements

\begin{verbatim}

    // Very readable with the bool funciton - if is single digit
    if (isSingleDigit(x))  
    {
        printf("x is little\n");
    } 
    else 
    {
        printf("x is big\n");
    }
    
\end{verbatim}

\section {Returning from {\tt main()}}

Now that we know functions that return values, we can look more closely at the 
return value of the {\tt main()} function.
It's supposed to return an integer:

\begin{verbatim}
  int main(void)
\end{verbatim}

The usual return value from {\tt main()} is 0, which indicates that
the program succeeded at whatever it was supposed to to.  If something
goes wrong, it is common to return -1, or some other value that
indicates what kind of error occurred.


\index{header file!stdlib.h}
\index{<stdlib.h>}
\index{EXIT\_SUCCESS}
\index{EXIT\_FAILURE}
The C standard library \texttt{<stdlib.h>} provides two predefined constants {\tt EXIT\_SUCCESS} and {\tt EXIT\_FAILURE}.
We can use these to return a descriptive result from our return statement. 


\begin{verbatim}

    #include <stdlib.h>
    int main (void)
    {
        return EXIT_SUCCESS; /* Program terminated successfully*/
    }
    
\end{verbatim}
%


Of course, you might wonder who this value gets returned to, since
we never call {\tt main()} ourselves.  It turns out that when the
operating system executes a program, it starts by calling {\tt main()}
in pretty much the same way it calls all the other functions.
When the program terminates it passes a value back 
that tells if the execution was successful or not. The operating system
can use this value to create error reports or even pass this value on 
to other programs.

There are even some parameters that can be passed to {\tt main()}
by the system, but we are not going to deal with them for a little
while, so we define  {\tt main()} as having no parameters:  {\tt int main (void)}. 

%\section {More recursion}
%\index{recursion}
%\index{language!complete}

%So far we have only learned a small subset of C, but you
%might be interested to know that this subset is now
%a {\bf complete} programming language, by which I
%mean that anything that can be computed can be expressed in this
%language.  Any program ever written could be rewritten
%using only the language features we have used so far (actually, we
%would need a few commands to control devices like the keyboard, mouse,
%disks, etc., but that's all).

%\index{Turing, Alan}

%Proving that claim is a non-trivial exercise first
%accomplished by Alan Turing, one of the first computer scientists
%(well, some would argue that he was a mathematician, but a lot of the
%early computer scientists started as mathematicians).  Accordingly, it
%is known as the Turing thesis.  If you take a course on the Theory of
%Computation, you will have a chance to see the proof.

%To give you an idea of what you can do with the tools we have learned
%so far, we'll evaluate a few recursively-defined
%mathematical functions.  A recursive definition is similar to a
%circular definition, in the sense that the definition contains a
%reference to the thing being defined.  A truly circular definition is
%typically not very useful:

%\begin{description}

%\item[frabjuous:] an adjective used to describe
%something that is frabjuous.

%\index{frabjuous}

%\end{description}

%If you saw that definition in the dictionary, you might be
%annoyed.  On the other hand, if you looked up the definition
%of the mathematical function {\bf factorial}, you might
%get something like:

%\begin{eqnarray*}
%&&  0! = 1 \\
%&&  n! = n \cdot (n-1)!
%\end{eqnarray*}

%(Factorial is usually denoted with the symbol $!$, which is
%not to be confused with the C logical operator {\tt !} which
%means NOT.)  This definition says that the factorial of 0 is 1,
%and the factorial of any other value, $n$, is $n$ multiplied
%by the factorial of $n-1$.  So $3!$ is 3 times $2!$, which is 2 times
%$1!$, which is 1 times 0!.  Putting it all together, we get
%$3!$ equal to 3 times 2 times 1 times 1, which is 6.

%If you can write a recursive definition of something, you can usually
%write a C program to evaluate it.  The first step is to decide what
%the parameters are for this function, and what the return type is.
%With a little thought, you should conclude that factorial takes an
%integer as a parameter and returns an integer:

%\begin{verbatim}
%  int factorial (int n)
%  {
%  }
%\end{verbatim}
%%
%If the argument happens to be zero, all we have to do is
%return 1:

%\begin{verbatim}
%  int Factorial (int n)
%  {
%      if (n == 0) 
%      {
%          return 1;
%      }
%  }
%\end{verbatim}
%%
%Otherwise, and this is the interesting part, we have to make
%a recursive call to find the factorial of $n-1$, and then
%multiply it by $n$.

%\begin{verbatim}
%  int Factorial (int n)
%  {
%      if (n == 0) 
%      {
%          return 1;
%      } 
%      else 
%      {
%          int recurse = Factorial (n-1);
%          int result = n * recurse;
%          return result;
%      }
%  }
%\end{verbatim}
%%
%If we look at the flow of execution for this program,
%it is similar to {\tt nLines} from the previous chapter.
%If we call {\tt factorial} with the value 3:

%Since 3 is not zero, we take the second branch and calculate
%the factorial of $n-1$...

%\begin{quote}
%Since 2 is not zero, we take the second branch and calculate
%the factorial of $n-1$...

%\begin{quote}
%Since 1 is not zero, we take the second branch and calculate
%the factorial of $n-1$...

%\begin{quote}
%Since 0 {\em is} zero, we take the first branch and return
%the value 1 immediately without making any more recursive
%calls.

%\end{quote}

%The return value (1) gets multiplied by {\tt n}, which is 1,
%and the result is returned.

%\end{quote}

%The return value (1) gets multiplied by {\tt n}, which is 2,
%and the result is returned.

%\end{quote}

%\noindent The return value (2) gets multiplied by {\tt n}, which is 3,
%and the result, 6, is returned to {\tt main}, or whoever
%called {\tt Factorial (3)}.

%\index{stack}
%\index{diagram!state}
%\index{diagram!stack}

%Here is what the stack diagram looks like for this sequence of
%function calls:

%\vspace{0.1in}
%\centerline{\epsfig{figure=figs/stack3.eps}}
%\vspace{0.1in}
%%
%The return values are shown being passed back up the stack.

%Notice that in the last instance of {\tt factorial}, the local
%variables {\tt recurse} and {\tt result} do not exist because
%when {\tt n=0} the branch that creates them does not execute.

%\section{Leap of faith}
%\index{leap of faith}

%Following the flow of execution is one way to read programs, but as
%you saw in the previous section, it can quickly become labarynthine.
%An alternative is what I call the ``leap of faith.''  When you come to
%a function call, instead of following the flow of execution, you
%{\em assume} that the function works correctly and returns the
%appropriate value.

%In fact, you are already practicing this leap of faith
%when you use built-in functions.  When you call {\tt cos}
%or {\tt exp}, you don't examine the implementations of 
%those functions.  You just assume that they work, because the people
%who wrote the built-in libraries were good programmers.

%Well, the same is true when you call one of your own functions.
%For example, in Section~\ref{bool} we wrote a function called
%{\tt IsSingleDigit} that determines whether a number is between
%0 and 9.  Once we have convinced ourselves that this function
%is correct---by testing and examination of the code---we can
%use the function without ever looking at the code again.

%The same is true of recursive programs.  When you get to the recursive
%call, instead of following the flow of execution, you should {\em
%assume} that the recursive call works (yields the correct result), and
%then ask yourself, ``Assuming that I can find the factorial of $n-1$,
%can I compute the factorial of $n$?''  In this case, it is clear that
%you can, by multiplying by $n$.

%Of course, it is a bit strange to assume that the function works
%correctly when you have not even finished writing it, but that's why
%it's called a leap of faith!

%\section{One more example}
%\index{factorial}

%In the previous example I used temporary variables to spell out the
%steps, and to make the code easier to debug, but I could have saved a
%few lines:

%\begin{verbatim}
%  int Factorial (int n) 
%  {
%      if (n == 0) 
%      {
%          return 1;
%      } 
%      else 
%      {
%          return n * Factorial (n-1);
%      }
%  }
%\end{verbatim}
%%
%From now on I will tend to use the more concise version, but
%I recommend that you use the more explicit version while you
%are developing code.   When you have it working, you can
%tighten it up, if you are feeling inspired.

%After {\tt Factorial}, the classic example of a recursively-defined
%mathematical function is {\tt Fibonacci}, which has the
%following definition:

%\begin{eqnarray*}
%&& fibonacci(0) = 1 \\
%&& fibonacci(1) = 1 \\
%&& fibonacci(n) = fibonacci(n-1) + fibonacci(n-2);
%\end{eqnarray*}
%%
%Translated into C, this is

%\begin{verbatim}
%  int Fibonacci (int n) 
%  {
%      if (n == 0 || n == 1) 
%      {
%          return 1;
%      } 
%      else 
%      {
%          return Fibonacci (n-1) + Fibonacci (n-2);
%      }
%  }
%\end{verbatim}
%%
%If you try to follow the flow of execution here, even for fairly small
%values of {\tt n}, your head explodes.  But according to the leap of
%faith, if we assume that the two recursive calls (yes, you can make
%two recursive calls) work correctly, then it is clear that we get the
%right result by adding them together.

\section{Glossary}

\begin{description}

\item[return type:]  The type of value a function returns.

\item[return value:]  The value provided as the result of a function
call.

\item[local variable:]  Also called a temporary variable, is a variable declared in a function and is only accessible from within the function in which it is declared

\item[dead code:]  Part of a program that can never be executed,
often because it appears after a {\tt return} statement.

\item[scaffolding:]  Code that is used during program development
but is not part of the final version.

\item[void:]  A special return type that indicates a void function;
that is, one that does not return a value.

\item[boolean:]  A value or variable that can take on one of
two states, often called $true$ and $false$.  In C, boolean
values are mainly stored in variables of type {\tt short} and 
preprocessor statements are used to define the states.

\item[flag:]  A variable that records
a condition or status information.

\item[comparison operator:]  An operator that compares two values
and produces a boolean that indicates the relationship between the
operands.

\item[logical operator:]  An operator that combines boolean values
in order to test compound conditions.

\index{return type}
\index{return value}
\index{dead code}
\index{scaffolding}
\index{void}
\index{bool}
\index{operator!conditional}
\index{operator!logical}

\end{description}

\section{Exercises}
\setcounter{exercisenum}{0}


% LaTeX source for textbook ``How to think like a computer scientist''
% Copyright (C) 1999  Allen B. Downey
% Copyright (C) 2009  Thomas Scheffler


\begin{exercise}
If you are given three sticks, you may or may not be able to arrange
them in a triangle.  For example, if one of the sticks is 12 inches
long and the other two are one inch long, it is clear that you will
not be able to get the short sticks to meet in the middle.  For any
three lengths, there is a simple test to see if it is possible to form
a triangle:

\begin{quotation}
``If any of the three lengths is greater than the sum of the other two,
then you cannot form a triangle.  Otherwise, you can.''
\end{quotation}

Write a function named {\tt isTriangle()} that it takes three integers as
arguments, and that returns either {\tt TRUE} or {\tt FALSE},
depending on whether you can or cannot form a triangle from sticks
with the given lengths.


The point of this exercise is to use conditional statements to
write a function that returns a value.
\end{exercise}


%%%%%%%%%%%%%%%%%%%%%%%%%%%%%%%%%%%%


%%%%%%%%%%%%%%%%%%%%%%%%%%%%%%%%%%%%

\begin{exercise}
What is the output of the following program?  Is there any dead code in this program?

The purpose of this exercise is to make sure you understand logical operators and the flow of execution through fruitful methods.

\begin{verbatim}

    #define TRUE 1
    #define FALSE 0
    
    short isHoopy(int);
    short isFrabjuous(int);
    
    int main(void) 
    {
        short flag1 = isHoopy(202);
        short flag2 = isFrabjuous(202);
        printf("%i\n", flag1);
        printf("%i\n", flag2);
        if (flag1 && flag2) 
        {
            puts("ping!");
        }
        if (flag1 || flag2) 
        {
            puts("pong!");
        }
        return EXIT_SUCCESS;
    }

    short isHoopy(int x)
    {
        short hoopyFlag;
        if (x % 2 == 0) 
        {
            hoopyFlag = TRUE;
        } 
        else 
        {
            hoopyFlag = FALSE;
        }
        return hoopyFlag;
    }
    
    short isFrabjuous(int x) 
    {
        short frabjuousFlag;
        if (x > 0) 
        {
            frabjuousFlag = TRUE;
        }
        else 
        {
            frabjuousFlag = FALSE;
        }
        return frabjuousFlag;
    }
     
\end{verbatim}
\end{exercise}



\begin{exercise}
\begin{enumerate}


\item Create a new program called {\tt Sum.c},
and type in the following two functions, their prototypes a main function.

\begin{verbatim}

    int functionOne(int m, int n) 
    {
        if (m == n) 
        {
            return n;
        } 
        else 
        {
            return m + functionOne(m + 1, n);
        }
    }
    
    int functionTwo(int m, int n) 
    {
        if (m == n) 
        {
            return n;
        } 
        else 
        {
            return n * functionTwo(m, n - 1);
        }
    }
    
\end{verbatim}
%
\item Write a few lines in {\tt main()} to test these functions.
Invoke them a couple of times, with a few different values,
and see what you get.  By some combination of testing and
examination of the code, figure out what these functions do,
and give them more meaningful names.  Add comments that
describe their function abstractly.

\item Add a {\tt prinf} statement to the beginning of both
functions so that they print their arguments each time they are
invoked.  This is a useful technique for debugging recursive
programs, since it demonstrates the flow of execution.

\end{enumerate}
\end{exercise}

\begin{exercise}
\label{ex.power}
Write a recursive function called {\tt power()} that
takes a double {\tt x} and an integer {\tt n} and that
returns $x^n$.  

Hint: a recursive definition of this
operation is {\tt power (x, n) = x * power (x, n-1)}.
Also, remember that anything raised to the zeroeth power
is 1.
\end{exercise}


%

\begin{exercise}
The distance between two points $(x_1, y_1)$ and $(x_2, y_2)$
is

\[Distance = \sqrt{(x_2 - x_1)^2 + (y_2 - y_1)^2} \]

Please write a function named {\tt distance()} that takes four
doubles as parameters---{\tt x1}, {\tt y1}, {\tt x2} and {\tt
y2}---and that prints the distance between the points.

You should first write a function called  {\tt sumSquares()}
that calculates and returns the sum of the squares of its arguments.
For example:

\begin{verbatim}
    double x = sumSquares (3.0, 4.0);
\end{verbatim}
%
would assign the value {\tt 25.0} to {\tt x}.

The point of this exercise is to write a new function that uses an
existing one.  You should first write the sumSquares function then use that function in your distance function.  Write a main function that tests each function
\end{exercise}


\begin{exercise}
The point of this exercise is to practice the syntax of fruitful
functions.

\begin{enumerate}

\item Use your existing solution to Exercise~\ref{ex.multadd} and make sure
you can still compile and run it.

\item Transform {\tt Multadd()} into a fruitful function, so
that instead of printing a result, it returns it.

\item Everywhere in the program that {\tt Multadd()} gets
invoked, change the invocation so that it stores the
result in a variable and/or prints the result.

\end{enumerate}
\end{exercise}




% LaTeX source for textbook ``How to think like a computer scientist''
% Copyright (C) 1999  Allen B. Downey
% Copyright (C) 2009  Thomas Scheffler
% Copyright (C) 2023  Michael Penta
\chapter{Iteration}

\section{More on assignments }
\index{assignment}
\index{statement!assignment}
\index{multiple assignment}
\index{self assignment}
\index{assignment!multiple}
\index{assignment!self}
\index{assignment!operator}
I haven't said much about it, but it is legal in C to
make more than one assignment to the same variable.  The
effect of the second assignment is to replace the old value
of the variable with a new value.

\begin{verbatim}
    int fred = 5;
    printf("%i", fred);
    fred = 7;
    printf("%i", fred);
\end{verbatim}
%
The output of this program is {\tt 57}, because the first
time we print {\tt fred} its value is 5, and the second time
its value is 7.

This kind of {\bf multiple assignment} is the reason I
described variables as a {\em container} for values.  When
you assign a value to a variable, you change the contents of
the container, as shown in the figure:

\vspace{0.1in}
\centerline{\includegraphics[height=1.8cm]{figs/assign2.pdf}}
\vspace{0.1in}

When there are multiple assignments to a variable, it is especially
important to distinguish between an assignment statement and a
statement of equality.  Because C uses the {\tt =} symbol for
assignment, it is tempting to interpret a statement like {\tt a = b}
as a statement of equality.  It is not! In many programming languages 
an alternate symbol is used
for assignment, such as {\tt <-} or {\tt :=}, in order to
avoid confusion.

First of all, equality is commutative, and assignment is not.
For example, in mathematics if $a = 7$ then $7 = a$.  But in
C the statement {\tt a = 7;} is legal, and {\tt 7 = a;}
is not.

Furthermore, in mathematics, a statement of equality is true
for all time.  If $a = b$ now, then $a$ will always equal $b$.
In C, an assignment statement can make two variables equal,
but they don't have to stay that way!

\begin{verbatim}
    int a = 5;
    int b = a; /* a and b no have the same value */
    a = 3;     /* a and b are no longer have the same value */
\end{verbatim}
%
The third line changes the value of {\tt a} but it does not
change the value of {\tt b}, and so they are no longer equal.

The ability to make multiple assignments to a variable means we can {\bf self assign} variables:

\begin{verbatim}
    int a = 5;  
    a = a + 1;       
\end{verbatim}
%
Assignment has the lowest precedent. 
This means that all other operations in the expression are evaluated before the assignment. 
In the first line code {\tt a} is initialized with a value of 5. In the second line, 
the expression  to the right of the assignment operator ($=$) is evaluated first. 
Since {\tt a} has the value 5, {\tt a + 1}  results in the value 6.
Finally we assign the result of 6 to {\tt a}..

\index{self assignment!operator}
\index{operator!self assignment}

Self assignment is actually very common. In fact, C has operators specifically for this task.
Several of these operators follow:

 \begin{verbatim}
 
 	  int x = 3;
 	  x += 2; /* Means x = x + 2 */      
 	  x *= 4; /* Means x = x * 4 */      
 	  x -= 6; /* Means x = x - 6 */      
 	  x /= 6; /* Means x = x / 6 */    
 	  x %= 2; /* Means x = x % 2 */ 
 	   
 \end{verbatim}
 %
For a very specific set of self assignment, we have an even shorter way to write it. 
When we add one to a number, we call that incrementing. 
Decrementing means subtracting one from a number.
Incrementing and decrementing are so common they have 
their own operators $++$ for increment and $--$ for decrementing.

\begin{verbatim}
 
    int x = 5;
    x++;               // Add 1 to x
    printf("%i\n", x); // Prints 6
    x--;               // Subtracts 1 from x
    printf("%i\n", x); // Prints 5
    
\end{verbatim}
%
The increment/decrement operator can be placed before the variable (pre-increment/pre-decrement) or after the variable (post-increment/post-decrement). When using the pre-operators, the value is modified then utilized in an expression. When the post-operators are used, the value is utilized then modified. 
\begin{verbatim}

    int x = 1;
    int y = 1;
    printf("%i\n", x++); // Prints 1, but x = 2 when completed
    printf("%i\n", ++y); // Prints 2 and y = 2 when completed
    printf("%i\n", x);   // Prints 2
    printf("%i\n", y);   // Prints 2
    
\end{verbatim}
 %
This behavior can be confusing and being careless with them can lead to errors. 
The best is to never mix the increment and decrement operators with other expressions. 
They should be executed on a line alone.
\begin{verbatim}

    int x = 1;
    int y = 1;
    x++;
    ++y;
    printf("%i\n", x); // Prints 2
    printf("%i\n", y); // Prints 2
    
\end{verbatim}
%
\section{Iteration}
\index{iteration}

One of the things computers are often used for is the automation
of repetitive tasks.  Repeating identical or similar tasks without
making errors is something that computers do well and people do
poorly.

In section \ref{recursion} we have seen programs that use {\bf recursion} to perform
repetition, such as {\tt printLines()} and {\tt countdown()}. 
I now want to introduce a new
type of repetition, that is called {\bf iteration}, and C provides
several language features that make it easier to write repetitive
programs.

We introduce two new control structures,  the {\tt while}
statement and the {\tt for} statement. 
These are repetition structures - they repeat stuff.

\section{The {\tt while} statement}
\index{statement!while}
\index{while statement}

We can write {\tt countdown()} program using a {\tt while} statement:

\begin{verbatim}

    void countdown(int n) 
    {
        while (n > 0) 
        {
            printf("%i\n", n);
            n = n - 1;
        }
        printf("Blastoff!\n");
    }
    
\end{verbatim}
%
You can almost read a {\tt while} statement as if it were
English.  What this means is, ``While {\tt n} is greater than
zero, continue displaying the value of {\tt n} and then reducing
the value of {\tt n} by 1.  When you get to zero, output the
word `Blastoff!'''

More formally, the flow of execution for a {\tt while} statement
is as follows:

 \begin{enumerate}
 	
 	\item Evaluate the condition in parentheses, yielding {\tt true}
or {\tt false}.

\item If the condition is false, exit the {\tt while} statement
and continue execution at the next statement.

\item If the condition is true, execute each of the statements
between the curly-brackets, and then go back to step 1.

\end{enumerate}

This type of flow is called a {\bf loop} because the third step loops
back around to the top.  Notice that if the condition is false the
first time through the loop, the statements inside the loop are
never executed.  The statements inside the loop are called
the {\bf body} of the loop.

\index{loop}
\index{loop!body}
\index{loop!infinite}
\index{body!loop}
\index{infinite loop}

The body of the loop should change the value of
one or more variables so that, eventually, the condition becomes
false and the loop terminates.  Otherwise the loop will repeat
forever, which is called an {\bf infinite loop}.  An endless
source of amusement for computer scientists is the observation
that the directions on shampoo, ``Lather, rinse, repeat,'' are
an infinite loop.

In the case of {\tt countdown()}, we can prove that the loop
will terminate because we know that the value of {\tt n} is
finite, and we can see that the value of {\tt n} gets smaller
each time through the loop (each {\bf iteration}), so
eventually we have to get to zero.  In other cases it is not
so easy to tell:

\begin{verbatim}

    void sequence(int n) 
    {
        while (n != 1) 
        {
            printf("%i\n", n);
            
            if (n % 2 == 0)    /* n is even */
            {          
                n = n / 2;
            } 
            else               /* n is odd */
            {                  
                n = n * 3 + 1;
            }
        }
    }
    
\end{verbatim}
%
The condition for this loop is {\tt n != 1}, so the loop
will continue until {\tt n} is 1, which will make the condition
false.

At each iteration, the program outputs the value of {\tt n} and then
checks whether it is even or odd.  If it is even, the value of
{\tt n} is divided by two.  If it is odd, the value is replaced
by $3n+1$.  For example, if the starting value (the argument passed
to {\tt Sequence}) is 3, the resulting sequence is
3, 10, 5, 16, 8, 4, 2, 1.

Since {\tt n} sometimes increases and sometimes decreases, there is no
obvious proof that {\tt n} will ever reach 1, or that the program will
terminate.  For some particular values of {\tt n}, we can prove
termination.  For example, if the starting value is a power of two, then
the value of {\tt n} will be even every time through the loop, until
we get to 1.  The previous example ends with such a sequence,
starting with 16.

Particular values aside, the interesting question is whether
we can prove that this program terminates for {\em all} values of n.
So far, no one has been able to prove it {\em or} disprove it!

\section{Looping Design Patterns}
When learning about loops it is helpful to identify common design pattern.
We will first discuss the a basic strategy for writing loops and then detail a number of common loop patterns such as the counting loop, the user termination loop, the sentinel loop, and the value validation loop.

The while loop is the most universal loop, but any while loop to work properly, it needs three things:
\begin{enumerate}
	\item A variable with an initial value before the loop, this will be the test variable in the condition
	\item A conditional test on the variable at the start of the loop
	\item A re-assignment on the test variable inside the loop body
\end{enumerate}

All of these components are needed for a loop to work. Let's identify these pieces from one of our previous loops. 

\begin{verbatim}

    void countdown(int n) 
    {
        // #1 n has an initial value when passed to the function
        while (n > 0) // #2 This is a test on the variable
        {
            printf("%i\n", n);
            n = n - 1;  // #3 This is an assignment on the variable
        }
	 	printf("Blastoff!\n");
    }
	
\end{verbatim}
%
\index{loops!counting}
Now that we have a basic strategy for laying out our while loops, let's examine some common patterns. The most common pattern we see it the {\bf counting loop}. This loop is used for iterating a specified number of times. If you can count the number of times you need your loop to run this may the pattern you need.

in a counting loop we start out counter with an initial value. Our condition will test if the count has yet to reach a specific limit. At each iteration we increment the count. 
\begin{verbatim}

    // #1 Count has an initial value of zero
    int count = 0;
    
    while (count  < 10) // #2 This is a test on the variable
    {
        printf ("%i\n", count); // Do something in the loop
        count++;                // #3 This is an assignment on the variable
    }
    
\end{verbatim}
%

\index{loops!sentinel}
Another common pattern we see it the {\bf sentinel loop}. A sentient is like a guard on watch. This loop will keep iterating until a special value is seen by the guard.  
\begin{verbatim}

    int value;
    puts("Enter a grade from 0 to 100. Enter a negative number to stop");
    scanf(" %i", &value); // #1 Get an intial value before the loop
    
    while (value >= 0) // #2 Test for a sentinel value
    {
        printf ("%i\n", value);  // Do something in the loop
        
        // #3 This is an assignment on the variable
        // We need to prompt and scan again
        puts("Enter a grade from 0 to 100. Enter a negative number to stop");
        scanf(" %i", &value);  // This is an assignment
    }
 	  
\end{verbatim}
%
\index{loops!user-terminated}
We can look at another sentinel, but we will call this a {\bf user terminated loop}. Like our sentinel, this uses user input to end a program.
\begin{verbatim}

    char choice;
    puts("Start a program y or n?");
    scanf(" %c", &choice); // #1 Get an intial value before the loop
    
    while (choice == 'y') // #2 Test for a sentinel value
    {
        // Run a whole program
        
        // #3 This is an assignment on the variable
        // We need to prompt and scan again
        
        puts("Do you want to run the progam again?");
        scanf(" %c", &choice); 
    }
    
\end{verbatim}
%
\index{loops!value validation}
This looping pattern is the value validation pattern. If you need a user to enter a number a bounded value. You can use this loop to ensure a value fits your bound.  In this loop we ask the user for input until an 'y' or an 'n' is entered. All three of our loop components are here, unlabeled. Can you identify them?
\begin{verbatim}

    char choice;
    puts("Start a program y or n?");
    scanf(" %c", &choice);
    
    while(choice != 'y' && choice != 'n')
    {
        puts("That is invalid input");
        puts("Start a program y or n");
        scanf(" %c", &choice);  
    }
    
    // More program after this
\end{verbatim}
%

\index{loops!accumulator}
Accumulators are loops that act to accumulate some value. Accumulator variables act in concert with loops. The variable must be initialized before the loop. It is not usually the control variable of the loop. It accumulates its value in the loop body. Here is an accumulator store the total sum of all the grades. 
\begin{verbatim}

    int sum = 0;
    int value;
    puts("Enter a grade from 0 to 100. Enter a negative number to stop");
    scanf(" %i", &value); // #1 Get an intial value before the loop
    
    while (value >= 0) // #2 Test for a sentinel value
    {
        sum = sum + value;
        
        // #3 This is an assignment on the variable
        // We need to prompt and scan again
        puts("Enter a grade from 0 to 100. Enter a negative number to stop");
        scanf(" %i", &value);  // This is an assignment
    }

    printf("%i\n", sum);
    
\end{verbatim}
%
\section{The for loop}
For loops are a short hand way of expressing a counting loop. It has all the same parts as a counting loop but it organizes them in a new way.

The for loop still needs these three things:
\begin{enumerate}
	\item A variable with an initial value before the loop, this will be the test variable in the condition
	\item A conditional test on the variable at the start of the loop
	\item A re-assignment on the test variable inside the loop body
\end{enumerate}
However in a for loop these are all written at the top of the structure.  

\begin{verbatim}
	  for ( intialization ; condition; assignment )
	  {
	  
	  }
\end{verbatim}
Below we present a counting while loop and its equivalent for loop
\begin{verbatim}

    // #1 An initial value
    int i = 0;
    while (i < 10) // #2 This is a test on the variable
    {
        printf("%i\n", i);
        
        i++; // #3 Re-assignment on the variable
    }
    
\end{verbatim}
%
\begin{verbatim}

    for (int i; i < 10; i++)  // #1, #2, #3
    {
        printf("%i\n", i);
    }

\end{verbatim}
%
As you can see a for loop is a more compact version of a counting loop, but where a while loop is a universal loop, a for loop is used primarily as a counting loop. For loops really show their worth when we work with arrays and strings. We will see more of them in upcoming chapters.
 
\section{Tables}
\index{table}
\index{logarithm}

One of the things loops are good for is generating
tabular data.  For example, before computers were readily available,
people had to calculate logarithms, sines and cosines, and other
common mathematical functions by hand.
To make that easier, there were books containing long tables
where you could find the values of various functions.
Creating these tables was slow and boring, and the result
tended to be full of errors.

When computers appeared on the scene, one of the initial reactions
was, ``This is great!  We can use the computers to generate the
tables, so there will be no errors.''  That turned out to be true
(mostly), but shortsighted.  Soon thereafter computers and
calculators were so pervasive that the tables became obsolete.

Well, almost.  It turns out that for some operations, computers
use tables of values to get an approximate answer, and then
perform computations to improve the approximation.  In some
cases, there have been errors in the underlying tables, most
famously in the table the original Intel Pentium used to perform
floating-point division.

\index{division!floating-point}

Although a ``log table'' is not as useful as it once was, it still
makes a good example of iteration.  The following program outputs a
sequence of values in the left column and their logarithms in the
right column:

\begin{verbatim}

    double x = 1.0;
    while (x < 10.0) 
    {
        printf("%.0f\t%f\n", x, log(x));
        x = x + 1.0;
    }
    
\end{verbatim}
%
The sequence \verb+\t+ represents a {\bf tab} character.
The sequence \verb+\n+ represents a newline character.  
They are so called \emph{escape sequences} which are used to encode
non-printable ASCII-characters.
Escape sequences can be included anywhere in a string, although in these examples
the sequence is the whole string.

A tab character causes the cursor to shift to the right until
it reaches one of the {\bf tab stops}, which are normally every
eight characters.  As we will see in a minute, tabs are useful
for making columns of text line up.
A newline character causes the cursor to move on to the next line.  
%Usually if a newline character appears by itself, I use {\tt endl}, but
%if it appears as part of a string, I use \verb+\n+.

The output of this program is:

\begin{verbatim}
    1      0.000000
    2      0.693147
    3      1.098612
    4      1.386294
    5      1.609438
    6      1.791759
    7      1.945910
    8      2.079442
    9      2.197225
\end{verbatim}
%
If these values seem odd, remember that the {\tt log()} function uses
base $e$.  Since powers of two are so important in computer science,
we often want to find logarithms with respect to base 2.  To do that,
we can use the following formula:

\[ \log_2 x = \frac {log_e x}{log_e 2} \]
%
Changing the output statement to

\begin{verbatim}
      printf("%.0f\t%f\n", x, log(x) / log(2.0));
\end{verbatim}
%
yields:

\begin{verbatim}
    1      0.000000
    2      1.000000
    3      1.584963
    4      2.000000
    5      2.321928
    6      2.584963
    7      2.807355
    8      3.000000
    9      3.169925
\end{verbatim}
%
We can see that 1, 2, 4 and 8 are powers of two, because
their logarithms base 2 are round numbers.  If we wanted to find
the logarithms of other powers of two, we could modify the
program like this:

\begin{verbatim}

    double x = 1.0;
    while (x < 100.0) 
    {
        printf("%.0f\t%.0f\n", x, log(x) / log(2.0));
        x = x * 2.0;
    }
    
\end{verbatim}
%
Now instead of adding something to {\tt x} each time through
the loop, which yields an arithmetic sequence, we multiply
{\tt x} by something, yielding a {\bf geometric} sequence.
The result is:

\begin{verbatim}
    1      0
    2      1
    4      2
    8      3
    16     4
    32     5
    64     6
\end{verbatim}
%
Because we are using tab characters between the columns, the
position of the second column does not depend on the number
of digits in the first column.

Log tables may not be useful any more, but for computer scientists,
knowing the powers of two is!  As an exercise, modify this program
so that it outputs the powers of two up to 65536
(that's $2^{16}$).  Print it out and memorize it.

\section{Two-dimensional tables}
\index{table!two-dimensional}

A two-dimensional table is a table where you choose a row and
a column and read the value at the intersection.  A multiplication
table is a good example.  Let's say you wanted to print a
multiplication table for the values from 1 to 6.

A good way to start is to write a simple loop that prints
the multiples of 2, all on one line.

\begin{verbatim}

    int i = 1;
    while (i <= 6) 
    {
        printf("%i   ", i * 2);
        i = i + 1;
    }
    printf("\n");
    
\end{verbatim}
%
The first line initializes a variable named {\tt i}, which is
going to act as a counter, or {\bf loop variable}.  As the
loop executes, the value of {\tt i} increases from 1 to 6,
and then when {\tt i} is 7, the loop terminates.  Each
time through the loop, we print the value {\tt 2*i} followed
by three spaces.  By omitting the  \verb+\n+ from the
first output statement, we get 
all the output on a single line.

\index{loop variable}
\index{variable!loop}

The output of this program is:

\begin{verbatim}
    2   4   6   8   10   12
\end{verbatim}
%
So far, so good.  The next step is to {\bf encapsulate} and {\bf
generalize}.

\section {Encapsulation and generalization}

Encapsulation usually means taking a piece of code and wrapping it up
in a function, allowing you to take advantage of all the things functions
are good for.  We have seen two examples of encapsulation, when we
wrote {\tt printParity()} in Section~\ref{alternative} and {\tt
isSingleDigit()} in Section~\ref{bool}.

Generalization means taking something specific, like printing
multiples of 2, and making it more general, like printing the
multiples of any integer.

\index{encapsulation}
\index{generalization}

Here's a function that encapsulates the loop from the previous
section and generalizes it to print multiples of {\tt n}.

\begin{verbatim}

    void printMultiples(int n)
    {
        int i = 1;
        while (i <= 6) 
        {
            printf("%i   ", i * n);
            i = i + 1;
        }
        printf("\n");
    }
    
\end{verbatim}
%
To encapsulate, all I had to do was add the first line,
which declares the name, parameter,
and return type.  To generalize, all I had to do was replace
the value 2 with the parameter {\tt n}.

If we call this function with the argument 2, we get the same
output as before.  With argument 3, the output is:

\begin{verbatim}
    3   6   9   12   15   18
\end{verbatim}
%
and with argument 4, the output is

\begin{verbatim}
    4   8   12   16   20   24 
\end{verbatim}
%
By now you can probably guess how we are going to print a
multiplication table: we'll call {\tt printMultiples()} repeatedly with
different arguments.  In fact, we are going to use another loop to
iterate through the rows.

\begin{verbatim}

    int i = 1;
    while (i <= 6) 
    {
        PrintMultiples (i);
        i = i + 1;
    }  
      
\end{verbatim}
%
First of all, notice how similar this loop is to the one inside {\tt
printMultiples()}.  I only replaced the call of the \texttt{printf()} function with 
the call of the \texttt{printMultiples()} function.

The output of this program is

\begin{verbatim}
    1   2   3   4   5   6   
    2   4   6   8   10   12   
    3   6   9   12   15   18   
    4   8   12   16   20   24   
    5   10   15   20   25   30   
    6   12   18   24   30   36   
\end{verbatim}
%
which is a (slightly sloppy) multiplication table.  If the
sloppiness bothers you, try replacing the spaces between
columns with tab characters {\tt ($\backslash$t)} and see what you get.

\section{Functions}
\index{function}

In the last section I mentioned ``all the things functions
are good for.''  About this time, you might be wondering
what exactly those things are.  Here are some of the reasons
functions are useful:

\begin{itemize}

\item By giving a name to a sequence of statements, you make
your program easier to read and debug.

\item Dividing a long program into functions allows you to
separate parts of the program, debug them in isolation, and
then compose them into a whole.

\item Functions facilitate both recursion and iteration.

\item Well-designed functions are often useful for many programs.
Once you write and debug one, you can reuse it.

\end{itemize}

\section{More encapsulation}
\index{encapsulation}
\index{program development!encapsulation}

To demonstrate encapsulation again, I'll take the code
from the previous section and wrap it up in a function:

\begin{verbatim}

    void printMultTable() 
    {
        int i = 1;
        while (i <= 6) 
        {
            PrintMultiples (i);
            i = i + 1;
        }
    }
    
\end{verbatim}
%
The process I am demonstrating is a common 
development plan.  You develop code gradually by adding
lines to {\tt main()} or someplace else, and then when you get
it working, you extract it and wrap it up in a function.

The reason this is useful is that you sometimes don't know
when you start writing exactly how to divide the program into
functions.  This approach lets you design as you go along.

\section{Local variables}

About this time, you might be wondering how we can use the same
variable {\tt i} in both {\tt printMultiples()} and {\tt
printMultTable()}.  Didn't I say that you can only declare a variable
once?  And doesn't it cause problems when one of the functions changes
the value of the variable?

The answer to both questions is ``no,'' because the {\tt i} in {\tt
printMultiples()} and the {\tt i} in {\tt printMultTable()} are
{\em not the same variable}.  They have the same name, but
they do not refer to the same storage location, and changing
the value of one of them has no effect on the other.

\index{local variable}
\index{variable!local}

Remember that variables that are declared inside a function definition
are local.  You cannot access a local variable from outside its
``home'' function, and you are free to have multiple variables with
the same name, as long as they are not in the same function.

The stack diagram for this program shows clearly that the
two variables named {\tt i} are not in the same storage location.
They can have different values, and changing one does not affect
the other.


\vspace{0.1in}
\centerline{\epsfig{figure=figs/stack4.pdf,width=6.5cm}}
\vspace{0.1in}
\index{stack diagram}
\index{diagram!stack}
%
Notice that the value of the parameter {\tt n} in
{\tt printMultiples()} has to be the same as the value
of {\tt i} in {\tt printMultTable()}.  On the other hand,
the value of {\tt i} in {\tt printMultiples()} goes
from 1 up to {\tt 6}.  In the diagram, it happens to be 3.
The next time through the loop it will be 4.

It is often a good idea to use different variable names in
different functions, to avoid confusion, but there are good
reasons to reuse names.  For example, it is common to
use the names {\tt i}, {\tt j} and {\tt k} as loop variables.
If you avoid using them in one function just because you
used them somewhere else, you will probably make the program
harder to read.

\index{loop variable}
\index{variable!loop}

%%
\section{More generalization}
\label{More generalization}
\index{generalization}

As another example of generalization, imagine you wanted
a program that would print a multiplication table of any
size, not just the 6x6 table.  You could add a parameter to
{\tt printMultTable()}:

\begin{verbatim}
    void printMultTable(int high) 
    {
        int i = 1;
        while (i <= high) 
        {
            PrintMultiples(i);
            i = i + 1;
        }
    }
\end{verbatim}
%
I replaced the value 6 with the parameter {\tt high}.  If I
call {\tt printMultTable()} with the argument 7, I get:

\begin{verbatim}
    1   2   3   4   5   6   
    2   4   6   8   10   12   
    3   6   9   12   15   18   
    4   8   12   16   20   24   
    5   10   15   20   25   30   
    6   12   18   24   30   36   
    7   14   21   28   35   42   
\end{verbatim}
%
which is fine, except that I probably want the table to
be square (same number of rows and columns), which means
I have to add another parameter to {\tt printMultiples()},
to specify how many columns the table should have.

Just to be annoying, I will also call this parameter {\tt high},
demonstrating that different functions can have parameters
with the same name (just like local variables):

\begin{verbatim}

    void PrintMultiples(int n, int high) 
    {
        int i = 1;
        while (i <= high) 
        {
            printf ("%i    ", n * i);
            i = i + 1;
        }    
        printf ("\n");
    }
    
    void PrintMultTable(int high) 
    {
        int i = 1;
        while (i <= high) 
        {
            PrintMultiples (i, high);
            i = i + 1;
        }
    }
    
\end{verbatim}
%
Notice that when I added a new parameter, I had to change the first
line of the function, and I also had to
change the place where the function is called in {\tt printMultTable()}.
As expected, this program generates a square 7x7 table:

\begin{verbatim}
    1   2   3   4   5   6   7   
    2   4   6   8   10   12   14   
    3   6   9   12   15   18   21   
    4   8   12   16   20   24   28   
    5   10   15   20   25   30   35   
    6   12   18   24   30   36   42   
    7   14   21   28   35   42   49
\end{verbatim}
%
When you generalize a function appropriately, you often find
that the resulting program has capabilities you did not intend.
For example, you might notice that the multiplication table
is symmetric, because $ab = ba$, so all the entries in the
table appear twice.  You could save ink by printing only
half the table.  To do that, you only have to change one
line of {\tt printMultTable()}.  Change

\begin{verbatim}
      printMultiples (i, high);
\end{verbatim}
%
to

\begin{verbatim}
      printMultiples (i, i);
\end{verbatim}
%
and you get:

\begin{verbatim}
    1   
    2   4   
    3   6   9   
    4   8   12   16   
    5   10   15   20   25   
    6   12   18   24   30   36   
    7   14   21   28   35   42   49  
\end{verbatim}
%
I'll leave it up to you to figure out how it works.

\section{Glossary}

\begin{description}

\item[loop:]  A statement that executes repeatedly while a
condition is true or until some condition is satisfied.

\item[infinite loop:]  A loop whose condition is always true.

\item[body:]  The statements inside the loop.

\item[iteration:]  One pass through (execution of) the body
of the loop, including the evaluation of the condition.

\item[tab:] A special character, written as \verb+\t+ in C,
that causes the cursor to move to the next tab stop on the
current line.

\item[encapsulate:]  To divide a large complex program into
components (like functions) and isolate the components from
each other (for example, by using local variables).

\item[local variable:]  A variable that is declared inside
a function and that exists only within that function.  Local variables
cannot be accessed from outside their home function, and do not
interfere with any other functions.

\item[generalize:]  To replace something unnecessarily specific
(like a constant value) with something appropriately general
(like a variable or parameter).  Generalization makes code more
versatile, more likely to be reused, and sometimes even easier
to write.

\item[development plan:]  A process for developing a program.
In this chapter, I demonstrated a style of development based on
developing code to do simple, specific things, and then encapsulating
and generalizing.

\index{loop}
\index{infinite loop}
\index{body}
\index{tab}
\index{loop!infinite}
\index{iteration}
\index{encapsulation}
\index{generalization}
\index{local variable}
\index{variable!local}
\index{program development}

\end{description}

\section{Exercises}
\setcounter{exercisenum}{0}

% LaTeX source for textbook ``How to think like a computer scientist''
% Copyright (C) 1999  Allen B. Downey
% Copyright (C) 2009  Thomas Scheffler

%%%%%%%%%%%%%%%%%%%%%%%%%%%%%%%%%%%%%%

\begin{exercise}\label{infloop}
%changed the condition on the loop so that it will terminate
%(was this *supposed* to be an infinite loop?)
\begin{verbatim}

    void loop(int n) 
    {
        int i = n;
        while (i > 1) 
        {
            printf("%i\n", i);
            if (i % 2 == 0) 
            {
                i = i/2;
            } 
            else 
            {
                i = i + 1;
            }
        }
    }

    int main (void) 
    {
        loop(10);
        
        return EXIT_SUCCESS;
    }
    
\end{verbatim}
%
\begin{enumerate}

\item  Draw a table that shows the value of the variables {\tt i} and {\tt n} during the execution of the program. 
The table should contain one column for each variable and one line for each iteration.


\item What is the output of this program?

\end{enumerate}
\end{exercise}

%%%%%%%%%%%%%%%%%%%%%%%%%%%%%%%%%%%%%%


\begin{exercise}
In Exercise~\ref{ex.power} we wrote a recursive version of {\tt
power()}, which takes a double {\tt x} and an integer {\tt n} and
returns $x^n$.  Now write an iterative function to perform the same
calculation.
\end{exercise}

%%%%%%%%%%%%%%%%%%%%%%%%%%%%%%%%%%%%%%



\begin{exercise}
	Define a function and prototype called getFirstNumber(). This function takes no parameters and returns an integer. The function should prompt a user for a number between 1 and 10. Use a loop to validate the input's value. If the value is invalid, continue to prompt until a valid value is received. When a valid value is given, it should be returned from the function
	
	Define a function and prototype called getSecondNumber(). This function takes one integer parameter and returns an integer. The parameter is a lower bound for a number. The function should prompt a user for a number between the lower bound and 15. Use a loop to validate the input's value. If the value is invalid, continue to prompt until a valid value is received. When a valid value is given, it should be returned from the function
	
	Define a function and prototype called printRange. This function takes two integer parameters, the lower and upper bound) and returns void. The function should use to print all the numbers form the lower to upper bound.
	
	Write a main program that calls the getFirstNumber function and uses the result as input to the getSecondNumber function. These will be the lower and upper bound for calling printRange, call print range with this input
	
	The goal of this exercise is to practice various loop patterns and practice using functions. 
\end{exercise}

%%%%%%%%%%%%%%%%%%%%%%%%%%%%%%%%%%%%%%


\begin{exercise}
	Write a program that meets the following description. 
	The user is asked if they want to play a game. If so the program should generate a random number between 1 and 10. This is the secret number. Prompt the user to make a guess at the secret number. Allow the to make 3 guesses. The game ends it the user guesses correctly or if they run out of guesses. Once the game is over, ask the user if they want to play again and replay the game.
	
	It is up to you to use the best loop designs for this problem. Be sure to use functions (can you reuse any functions you previously wrote)
	
	The goal of this exercise is to practice program design and loop types.
\end{exercise}

%%%%%%%%%%%%%%%%%%%%%%%%%%%%%%%%%%%%%%



\include{Chapter7_Array}
\include{Chapter8_String}
\include{Chapter9_Struct}
\include{Chapter10_Files}


\appendix
\include{Append1}
\include{ASCII}
\include{Formats}
\printindex

\tracingmacros=0
\end{document}




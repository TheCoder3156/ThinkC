% LaTeX source for textbook ``How to think like a computer scientist''
% Copyright (C) 1999  Allen B. Downey
% Copyright (C) 2009  Thomas Scheffler

\begin{exercise}
Use the variables initialized below to evaluate each expression. State if C would consider the resulting value as true or false.
\begin{verbatim}
    int m = 0;
    int n = 1;
    char p = 'k';
    char s = 'F';
    double t = 1.2;   
\end{verbatim}
\begin{enumerate}
	\item {\tt m / n}
	\item {\tt m + n}
	\item {\tt m}
	\item {\tt p <= s}
	\item {\tt t < n}
\end{enumerate}
\end{exercise}

\begin{exercise}
This exercise reviews the flow of execution through a program
with multiple methods.  Read the following code and answer the
questions below.

\begin{verbatim}

    #include <stdio.h>
    #include <stdlib.h>
    
    void zippo (int, int);
    void baffle (int);
    
    int main (void)
    {
        zippo(5, 13);
        
        return EXIT_SUCCESS;
    }
    
    void baffle (int output)
    {
        printf("%i\n", output);
        zippo(12, -5);
    }
    
    void zippo (int quince, int flag)
    {
        if (flag < 0)
        {
            printf("%i zoop\n", quince);
        }
        else
        {
            printf("rattle ");
            baffle(quince);
            printf("boo-wa-ha-ha\n");
        }
    }
    
\end{verbatim}
%
\begin{enumerate}

\item Write the number {\tt 1} next to the first {\em statement}
of this program that will be executed.  Be careful to distinguish
things that are statements from things that are not.

\item Write the number {\tt 2} next to the second statement, and so on
until the end of the program.  If a statement is executed more than
once, it might end up with more than one number next to it.

\item What is the value of the parameter {\tt quince} when {\tt baffle()}
gets invoked for the first time?

\item What is the exact output of this program? Pay close attention to the printed white space like spaces, tabs, and new lines.

\end{enumerate}

\end{exercise}

\begin{exercise}
	In this exercise you will practice using random with functions
	
		\begin{enumerate}
		\item Define a function and a prototype called rollDie that has no parameters. 
	In the function call rand to generate a random number 1, 2, 3, 4, 5, or 6. Print the resulting value.
	
	\item Define a main function that seeds the random with epoch time (use time() from time.h) and calls your function 3 times
		\end{enumerate}
\end{exercise}

\begin{exercise}
	In this exercise you will practice using selection statements with functions
	\begin{enumerate}
		\item Define a function and a prototype called validate that has one parameter, an int. 
		If the int is 0 the function should print an error message and terminate. 
		If the parameter is non-zero, multiply the value by 2 and print the result.
		
		\item 
		Define a main function that calls your function 3 times with the following arguments: 0, 3, -2
		
	\end{enumerate}

\end{exercise}


\begin{exercise}
There is an old song about 
beer bottles that can be expressed recursively.

The first verse of the song ``99 Bottles of Beer'' is:

\begin{quote}
99 bottles of beer on the wall,
99 bottles of beer,
ya' take one down, ya' pass it around,
98 bottles of beer on the wall.
\end{quote}

Subsequent verses are identical except that the number
of bottles gets smaller by one in each verse, until the
last verse:

\begin{quote}
No bottles of beer on the wall,
no bottles of beer,
ya' can't take one down, ya' can't pass it around,
'cause there are no more bottles of beer on the wall!
\end{quote}
%
And then the song (finally) ends.

Write a program that prints the entire lyrics of
``99 Bottles of Beer.''  Your program should include a
recursive method that does the hard part, but you also
might want to write additional methods to separate the major
functions of the program.

The last verse, when the number of bottles left is 0, is the base case. 
The other verses are the recursive step. 

As you are developing your code, you will probably
want to test it with a small number of verses, like
``3 Bottles of Beer.''

The purpose of this exercise is to take a problem and break it
into smaller problems, and to solve the smaller problems by writing
simple, easily-debugged methods.
\end{exercise}


\begin{exercise}
You can use the  {\tt getchar()} function in C to
get character input from the user through the keyboard.
This function stops the execution of the program and waits
for the input from the user. 

The {\tt getchar()} function has the type {\tt int} and does
not require an argument. It returns the ASCII-Code (cf. Appendix~\ref{ASCII-Table})
of the key that has been pressed on the keyboard.
\begin{enumerate}
\item Write a program, that asks the user 
to input a digit between  0 and 9. 

\item Test the input from the user and display an error message 
if the returned value is not a digit. The program should then
be terminated.
If the test is successful, the program should print the 
input value on the computer screen.
\end{enumerate}


\end{exercise}

% Kapitel 5 (Return)
%Schreiben Sie dazu eine Funktion {\tt AsciiToNumber()} welche ein
%{\tt int} als Typ und als Argument besitzt. �bergeben Sie der Funktion
%den eingelesenen Wert und 




% String--Ausgabe!

%\begin{exercise}
%Lesen Sie das folgende Programm. Notieren Sie die Ausgabe die w�hrend der
%Abarbeitung des Programms erzeugt wird.

%\begin{verbatim}
%  void Zoop (String fred, int bob) 
%  {
%        printf ("%s\n", fred);
%        if (bob == 5) 
%        {
%            ping ("not ");
%        } 
%        else 
%        {
%            printf ("!\n");
%        }
%    }

%  int main (void) 
%  {
%        int bizz = 5;
%        int buzz = 2;
%        zoop ("just for", bizz);
%        clink (2*buzz);
%    }

%  void clink (int fork) 
%  {
%        printf ("It's ");
%        zoop ("breakfast ", fork) ;
%  }

%  void ping (String strangStrung) 
%  {
%        printf ("any %smore \n", strangStrung);
%    }
%}
%\end{verbatim}
%\end{exercise}



